\chapter{Appendix}

\section{Useful tutorials and documentations}

In order to learn more about ARC several other tutorials and documentations have been written:
\begin{itemize}
 \item ARC Web Services Quick Usage Guide~\cite{2008_UNKNOWN}
 \item The Hosting Environment of the Advanced Resource Connector middleware~\cite{2008_Cameron}.
 \item Security framework of ARC1~\cite{QIANG_2008}.
 \item Documentation of the ARC storage system~\cite{Nagy_2008}.
 \item Eclipse WTP 1.5.1, Introduction to the WSDL Editor, \url{http://wiki.eclipse.org/index.php/Introduction_to_the_WSDL_Editor}


\end{itemize}




\section{Important XSD files}\label{sec:impXSD}

\lstsetJUSTXML
\lstinputlisting
	[
	label=lst:mcc.xsd,
	caption={[XML schema of configuration files.]\textbf{XML schema of configuration files.}}
	]{tex_appendix/loader.xsd}

\lstsetJUSTXML
\lstinputlisting
	[
	label=lst:tcp.xsd,
	caption={[XML schema of the TCP component.]\textbf{XML schema of the TCP component.}}
	]{tex_appendix/tcp.xsd}

\lstsetJUSTXML
\lstinputlisting
	[
	label=lst:tls.xsd,
	caption={[XML schema of the TLS component.]\textbf{XML schema of  the TLS component.}}
	]{tex_appendix/tls.xsd}

\lstsetJUSTXML
\lstinputlisting
	[
	label=lst:http.xsd,
	caption={[XML schema of the HTTP component.]\textbf{XML schema of the HTTP component.}}
	]{tex_appendix/http.xsd}









% 
% Questions:
% \begin{itemize}
%  \item The images of the HED structure always are ordered like: TCP - TLS - HTTP - PLEXER - SOAP - Web Service   but the source code examples are: TCP - TLS - HTTP - SOAP - PLEXER - Web Service\\
% --- ANSWER BY WEIZHONG: In my understanding (because plexer is not my initial idea), plexer (as a sort of hub) can be put at both after http or soap. If you put after soap,then you are doing switching for different web services. While I put it after http, because somehow, I need to switch the message after http-processing, in more detail, one branch for soap and then web service/s,
%  one branch for a non-soap service/server (the SP Service is not a service based on SOAP, actually it is based on http, so it is kind of web application).
% 
% \item The HED does not act as a front end server which delegates its task to new instances acting on a diffrent port such that the daemon is always available. 
% \end{itemize}

% 
% 
 \task{RENAMING HED configuration file into~ARC service configuration file - proposed by Marek}
 \task{(echo service) client configuration file - proposed by Marek}
 \task{Check commas using \url{http://leo.stcloudstate.edu/punct/comma.html}}

 \task{Bisher noch nicht beachtet: \\
\\\textcolor{urgent}{
In general, a comma is used when the
subordinate clause precedes the main clause, like this:\\
\\
``When subjects were aware of the purpose of the experiment, no effect was
observed.''\\
\\
Furthermore:\\
\\
      ``We discarded specimens that had become contaminated.''\\
\\
In this case, the relative clause is restrictive because it defines exactly which specimens
were discarded. Restrictive clauses should not be preceded by a comma.\\
\\
  In contrast, a non-restrictive clause gives additional information about an object that
has already been identified; for example:\\
\\
      ``The file server uses the XFS file system, which supports journaling and pro-
      vides excellent performance.''\\
\\
In this case, the relative clause does not restrict or define the particular file system being
talked about, but merely provides additional information. Non-restrictive clauses should
be preceded by a comma.}}



