\section{Examples of \ref{ainit}}
\subsection{Importing the Logger class from module ''logger'' of ''arcom'' package}
(Example \ref{loglog})
\label{cloglog}
\begin{verbatim}
import arcom
# From the logger module within the arcom package
# import the Logger class
Logger = arcom.import_class_from_string('arcom.logger.Logger')
# Now the class could be access through Logger
Logger
\end{verbatim}

\subsection{Getting attributes from an XMLNode}
(Example \ref{getattr})
\label{cgetattr}
\begin{verbatim}
import arc
import arcom
# Create XMLNode
n = arc.XMLNode(arc.NS({'me':'http://example.com/myExample'}),'me:myExample')
# Add attributes to node and set their values
n.NewAttribute('foo').Set('Hello')
n.NewAttribute('moo').Set('World')
# Show XML
n.GetXML()
# get attributes of this XMLNode
a = arcom.get_attributes(n)
# result is a dictionary
type(a)
# show dictionary
a
# extracting an attribute value
a['moo']
\end{verbatim}

\subsection{Getting child nodes}
(Example \ref{getchld})
\label{cgetchld}
\begin{verbatim}
import arc
import arcom
# Create root node
a = arc.XMLNode(arc.NS(),'a')
# Create child nodes for 'a'
b = a.NewChild('b')
c = a.NewChild('c')
d = a.NewChild('d')
# Create child nodes for 'c'
e = c.NewChild('e')
f = c.NewChild('f')
# Create child node for 'd'
g = d.NewChild('g')
# show XML
a.GetXML()
# get child nodes for 'a'
tmp = arcom.get_child_nodes(a)
# result is a list
type(tmp)
# 'a' has 3 children
len(tmp)
# show name for each
', '.join(x.Name() for x in tmp)
# 'b' has no children
tmp = arcom.get_child_nodes(b)
len(tmp)
# 'c' has 2 children: 'e' and 'f'
tmp = arcom.get_child_nodes(c)
len(tmp)
', '.join(x.Name() for x in tmp)
# 'd' has 1 child: 'g'
tmp = arcom.get_child_nodes(d)
len(tmp)
', '.join(x.Name() for x in tmp)
# nodes 'e', 'f' and 'g' have no children
tmp = arcom.get_child_nodes(e)
len(tmp)
tmp = arcom.get_child_nodes(f)
len(tmp)
tmp = arcom.get_child_nodes(g)
len(tmp)
\end{verbatim}

\subsection{Get values of specified children}
(Example \ref{getchdval})
\label{cgetchdval}
\begin{verbatim}
import arc
import arcom
# Create XMLNode
n = arc.XMLNode(arc.NS(),'node')
# Create 3 child nodes (two of which get the same name)
x = n.NewChild('same')
y = n.NewChild('same')
z = n.NewChild('different')
# Set values for nodes
x.Set('firstEQ')
y.Set('secondEQ')
z.Set('DIFF')
# Show XML
n.GetXML()
# Get child values where name is 'same'
tmp = arcom.get_child_values_by_name(n,'same')
# Result is a list
type(tmp)
# Show result
tmp
# Get child values where name is 'different'
tmp = arcom.get_child_values_by_name(n,'different')
# Show result
tmp
\end{verbatim}

\subsection{Creating DataPoint from URL}
(Example \ref{url2dp})
\label{curl2dp}
\begin{verbatim}
import arc
import arcom
tmpList = []
status = ''
# create DataPoint from a local directory
dp = arcom.datapoint_from_url('file:///usr/local/share/arc')
# list files
(files, stat) = dp.ListFiles()
# if it is not empty
if files:
    status = 'found'
    # for all the entries, get type and name
    for f in files:
        if (f.GetType() == arc.FileInfo.file_type_file):
            type = 'file'
        elif (f.GetType() == arc.FileInfo.file_type_dir):
            type = 'dir'
        else:
            type = 'unknown'
        # get results together in a list
        tmpList.append(f.GetName() + ' (' + type + ')\n')
else:
    status = 'Could not access data. Reason: %s' % str(stat)

# see result
str(stat)
# show list
tmpList
\end{verbatim}

\subsection{Parsing a URL}
(Example \ref{parseurl})
\label{cparseurl}
\begin{verbatim}
import arcom
proto, host, port, path = arcom.parse_url('boo://no.one.here:123/foo')
proto
host
port
path
\end{verbatim}

