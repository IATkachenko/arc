\section{Examples of \ref{aservice}}
\subsection{Using DummyService - an example service based on arcom.service.Service}
(Example \ref{srvex})
\label{csrvex}
\begin{verbatim}
# Reminder:
# import arc
#
# class Dummy:
#   def foo(self, foomsg = ''):
#     return ''.join(['Foo message is: ',str(foomsg)])
# 
# from arcom.service import Service
# 
# class DummyService(Service):
#   """ DummyService class based on Service class in arcom.service """
#   ...

# Create instance of DummyService
ds = DummyService(None)
# Create namespace - it will be used for the message sent
ns = arc.NS({'dmmy':'urn:foo'})
# Create SOAP Envelope
se = arc.SOAPEnvelope(ns)
# Create content
se.NewChild('dmmy:foo').Set('Foo!')
# Create SOAP Message
msg = arc.SOAPMessage()
# Payload
pls = arc.PayloadSOAP(se)
# Set Message Payload
msg.Payload(pls)
# create outgoing message that will hold response
outse = arc.SOAPEnvelope(ns)
outmsg = arc.SOAPMessage()
outpls = arc.PayloadSOAP(outse)
outmsg.Payload(outpls)
# call service
ds.process(msg, outmsg)
# show result
print outmsg.Payload().GetXML()
\end{verbatim}

\subsection{Retrieving state of DummyService}
(Example \ref{RetDSstate})
\label{cRetDSstate}
\begin{verbatim}
# Reminder:
# import arc
#
# ...
# 
# from arcom.service import Service
# 
# class DummyService(Service):
#   """ DummyService class based on Service class in arcom.service """
# 
#   ...
# 
#   def status(self, inpayload):
#       # get local info
#       fooinfo = self.GetLocalInformation()
#       
#       # get service status
#       foostatus = str(fooinfo.Get('AdminDomain').Get('Services').Get('Service').\
#       Get('Endpoint').Get('ServingState'))
#       ...

# Create instance of DummyService
ds = DummyService(None)
# Create namespace - it will be used for the message sent
ns = arc.NS({'dmmy':'urn:foo'})
# Create SOAP Envelope
se = arc.SOAPEnvelope(ns)
# Create content
se.NewChild('dmmy:status')
# Create SOAP Message
msg = arc.SOAPMessage()
# Payload
pls = arc.PayloadSOAP(se)
# Set Message Payload
msg.Payload(pls)
# create outgoing message that will hold response
outse = arc.SOAPEnvelope(ns)
outmsg = arc.SOAPMessage()
outpls = arc.PayloadSOAP(outse)
outmsg.Payload(outpls)
# call service
ds.process(msg, outmsg)
# show result
print outmsg.Payload().GetXML()
\end{verbatim}

\subsection{parse\_node}
(Example \ref{pnex})
\label{cpnex}
\begin{verbatim}
from arc import XMLNode
from arcom.service import parse_node

# create xml node
xml = XMLNode('''
<myList>
  <myElement>
    <myID>0</myID>
    <LN>/</LN>
  </myElement>
  <myElement>
    <myID>1</myID>
    <LN>/testfile</LN>
  </myElement>
</myList>''')

# default: single = False and string = True
parse_node(xml, ['myID','LN'])
# single = True
parse_node(xml, ['myID','LN'], single = True)
parse_node(xml, ['myID','LN'], True)
# string = False
parse_node(xml, ['myID','LN'], string = False)
\end{verbatim}

\subsection{parse\_to\_dict}
(Example \ref{p2dex})
\label{cp2dex}
\begin{verbatim}
from arc import XMLNode
from arcom.service import parse_to_dict

# create xml node
xml = XMLNode('''
<myList>
  <myElement>
    <myID>123</myID>
    <refID>abc</refID>
    <state>alive</state>
    <size>123456</size>
  </myElement>
  <myElement>
    <myID>456</myID>
    <refID>fed</refID>
    <state>alive</state>
    <size>987</size>
  </myElement>
</myList>''')

# example 1 - 'myID' as key
# 'state' and 'size' goes to the inner dictionary
parse_to_dict(xml, ['myID','state','size'])

# example 2 - 'myID' as key
# 'myID' also goes to the inner dictionary
parse_to_dict(xml, ['myID','myID','state','size'])
\end{verbatim}

\subsection{create\_response}
(Example \ref{crespex})
\label{ccrespex}
\begin{verbatim}
import arc
from arcom.service import create_response

method_name = 'dummy'
# create tag_names; 'myID' will be the key
tag_names = ['myID', 'state', 'size']
# create elements
# elements of the lists in this dictionary are values and will be tagged 
# according to names found in tag_names except for the first element
# e.g.: 'alive' will be tagged with 'state'
elements = {'123': ['alive', '123456'], '456': ['alive', '987']}
# create empty payload
payload = arc.PayloadSOAP(arc.NS())

# create response
response = create_response(method_name, tag_names, elements, payload)

# show it
print response.GetXML(True)
\end{verbatim}

\subsection{node\_to\_data - 1}
(Example \ref{n2dex1})
\label{cn2dex1}
\begin{verbatim}
from arc import XMLNode
from arcom.service import node_to_data

# create node
xml = XMLNode('''
<myElement>
  <myID>0</myID>
  <refID>abc</refID>
  <section>states</section>
  <property>spares</property>
  <value>2</value>
  <dummy>foo</dummy>
  <changeType>set</changeType>
</myElement>
''')

# create names
# 'myID' will be the key
# all the other names will be in a list in the specified order
# 'dummy' will be left out
names = ['myID', 'refID', 'changeType', 'section', 'property', 'value']

# call node_to_data
node_to_data(xml,names)
\end{verbatim}

\subsection{node\_to\_data - 2}
(Example \ref{n2dex2})
\label{cn2dex2}
\begin{verbatim}
from arc import XMLNode
from arcom.service import node_to_data

# create node
xml = XMLNode('''
<getRequest>
  <GUID>11</GUID>
  <myID>99</myID>
</getRequest>
''')

# create names
# 'myID' will be the key
names = ['myID', 'GUID']

# call node_to_data
# GUID is the single value so do not put it in a list
node_to_data(xml,names, True)
\end{verbatim}

\subsection{get\_data\_node}
(Example \ref{gdnex})
\label{cgdnex}
\begin{verbatim}
from arc import XMLNode
from arcom.service import get_data_node

# create node
xml = XMLNode('''
<Envelope>
  <Body>
    <method>
      <data>Value</data>
    </method>
  </Body>
</Envelope>
''')

# call get_data_node
dn = get_data_node(xml)

# show result
dn.GetXML()
\end{verbatim}

