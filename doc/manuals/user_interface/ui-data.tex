\subsection{arcls}\label{sec:arcls}
\texttt{arcls}\index{arcls}\index{commands:arcls} is a simple
utility that allows to list contents and view some attributes of
objects of a specified (by a URL) remote directory.

\hspace*{0.5cm}
\begin{shaded}
   \uicommand{arcls [options] $<$URL$>$}
\end{shaded}
\versions{ARC 0.9}
\begin{longtable}{llp{8cm}}
    Options:&&\\
    \texttt{-h} && short help\\
    \texttt{-v} && print version information\\
    \texttt{-d} & \textit{debuglevel} &debug level is one of  FATAL, ERROR, WARNING, INFO, VERBOSE or DEBUG\\
    \texttt{-l} &  & detailed listing\\
    \texttt{-L} &  & detailed listing including URLs from which file can 
    be downloaded\\
    \texttt{-m} && display all available metadata\\
    Arguments:&&\\
    \texttt{URL} && file or directory URL\\
\end{longtable}

This tool is very convenient not only because it allows to list files
at a Storage Element or records in an indexing service, but also
because it can give a quick overview of a job's working directory,
which is explicitly given by job ID.

Usage examples can be as follows:

\begin{verbatim}
    arcls -L rls://rls.nordugrid.org:38203/logical_file_name
    arcls -l gsiftp://lscf.nbi.dk:2811/jobs/1323842831451666535
    arcls srm://grid.uio.no:8446/srm/managerv2?SFN=/johndoe/log2
\end{verbatim}

Examples of URLs accepted by this tool can be found in
Section~\ref{sec:urls}, though \texttt{arcls} won't be able to list a
directory at an HTTP server, as they normally do not return directory
listings.

\subsection{arccp}\label{sec:arccp}

\texttt{arccp}\index{arccp}\index{commands:arccp} is a powerful
tool to copy files over the Grid. It is a part of the A-REX,
but can be used by the User Interface as well.
\hspace*{0.5cm}
\begin{shaded}
   \uicommand{arccp [options] $<$source$>$ $<$destination$>$}
\end{shaded}
\versions{ARC 0.9}
\begin{longtable}{llp{8cm}}
    Options:&&\\
    \texttt{-h} && short help\\
    \texttt{-v} && print version information\\
    \texttt{-d} & \textit{debuglevel} &debug level is one of  FATAL, ERROR, WARNING, INFO, VERBOSE or DEBUG\\
    \texttt{-y} & \textit{cache\_path} & path to local cache (use to put file into cache)\\
    \texttt{-p} && use passive transfer (does not work if secure is on, default if secure is not requested)\\
    \texttt{-n} && do not try to force passive transfer\\
    \texttt{-i} && show progress indicator\\
    \texttt{-u} && use secure transfer (insecure by default)\\
    \texttt{-r} & \textit{recursion\_level} & operate recursively (if possible) up to specified level (0 - no recursion)\\
    \texttt{-R} & \textit{number} & how many times to retry transfer of every file before failing\\
    \texttt{-t} & \textit{time} & timeout in seconds (default 20)\\
    \texttt{-f} && if the destination is an indexing service and not the same as the source and the destination is already registered, then the copy is normally not done. However, if this option is specified the source is assumed to be a replica of the destination created in an uncontrolled way and the copy is done like in case of replication\\
    \texttt{-T} && do not transfer file, just register it - destination must be non-existing meta-url\\
    Arguments:&&\\
    \texttt{source} && source URL\\
    \texttt{destination} && destination URL\\
\end{longtable}

This command transfers contents of a file between 2 end-points.
End-points are represented by URLs or meta-URLs. For supported
endpoints please refer to Section~\ref{sec:urls}.

\texttt{arccp} can perform multi-stream transfers if \texttt{threads}
URL option is specified and server supports it.

Source URL can end with \verb#"/"#. In that case, the whole fileset
(directory) will be copied. Also, if the destination ends with
\verb#"/"#, it is extended with part of source URL after last
\verb#"/"#, thus allowing users to skip the destination file or
directory name if it is meant to be identical to the source.

Usage examples of \texttt{arccp} are:

\begin{verbatim}
    arccp gsiftp://lscf.nbi.dk:2811/jobs/1323842831451666535/job.out \
              file:///home/myname/job2.out
    arccp gsiftp://aftpexp.bnl.gov;threads=10/rep/my.file \
              rls://grid.uio.no/zebra4.f
    arccp http://www.nordugrid.org/data/somefile gsiftp://hathi.hep.lu.se/data/
\end{verbatim}

\subsection{arcrm}\label{sec:arcrm}

The \texttt{arcrm}\index{arcrm}\index{commands:arcrm}
command allows users to erase files at any location specified by a
valid URL.
\hspace*{0.5cm}
\begin{shaded}
   \uicommand{arcrm [options] $<$source$>$}
\end{shaded}
\versions{ARC 0.9}
\begin{longtable}{llp{8cm}}
    Options:&&\\
    \texttt{-h} && short help\\
    \texttt{-v} && print version information\\
    \texttt{-d} & \textit{debuglevel} &debug level is one of  FATAL, ERROR, WARNING, INFO, VERBOSE or DEBUG\\
    \texttt{-c} & &continue with meta-data even if it failed to delete real file\\
    Arguments:&&\\
    \texttt{source} && source URL\\
\end{longtable}

\begin{framed}
   A convenient use for \texttt{arcrm} is to erase the files in a data
   indexing catalog (LFC, RLS or such), as it will not only remove the
   physical instance, but also will clean up the database record.
\end{framed}

Here is an \texttt{arcrm} example:

\begin{verbatim}
    arcrm lfc://grid.uio.no/grid/atlas/AOD_0947.pool.root
\end{verbatim}

\subsection{arcacl}\label{sec:arcacl}

\index{arcacl}\index{commands:arcacl}This command retrieves or modifies
access control information associated with a stored object if service
supports GridSite GACL language~\cite{gacl} for access control.
\hspace*{0.5cm}
\begin{shaded}
   \uicommand{arcacl [options] get$|$put $<$URL$>$}
\end{shaded}
\versions{ARC 0.9}
\begin{longtable}{llp{8cm}}
   Options:&&\\
    \texttt{-d, -debug} & \textit{debuglevel} &debug level is one of  FATAL, ERROR, WARNING, INFO, VERBOSE or DEBUG\\
    \texttt{-v} && print version information\\
    \texttt{-h} && short help\\
   Arguments:&&\\
    \texttt{get} &\textit{URL}& get Grid ACL for the object\\
    \texttt{put} &\textit{URL}& set Grid ACL for the object\\
    \texttt{URL} && object URL; curently only gsiftp and sse URLs are supported\\
\end{longtable}

The ACL document (an XML file) is printed to standard output when
\texttt{get} is requested, and is acquired from standard input when
\texttt{set} is specified\footnote{In ARC $\leq$ 0.5.28, \texttt{set}
  was used instead of \texttt{put}}. Usage examples are:
\begin{verbatim}
    arcacl get gsiftp://se1.ndgf.csc.fi/ndgf/tutorial/dirname/filename
    arcacl set gsiftp://se1.ndgf.csc.fi/ndgf/tutorial/dirname/filename < myacl
\end{verbatim}

\subsection{arctransfer}\label{sec:arctransfer}
\index{arctransfer}\index{commands:arctransfer}
The \texttt{arctransfer} command is not implemented.

