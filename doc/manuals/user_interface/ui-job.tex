\subsection{arcsub}\label{sec:arcsub}
The \texttt{arcsub}\index{arcsub}\index{commands:arcsub} command is the
most essential one, as it is used for submitting jobs to the Grid
resources. \index{submit job}\texttt{arcsub} matches user's job
description to the information collected from the Grid, and the
optimal site is being selected for job submission. The job description
is then being forwarded to that site, in order to be submitted to the
Local Resource Management System (LRMS), which can be, e.g., PBS or
Condor or SGE etc.

\hspace*{0.5cm}
\begin{shaded}
   \uicommand{arcsub [options] [filename ...]}
\end{shaded}
\versions{ARC 0.9}
\begin{longtable}{llp{8cm}}
   Options:&&\\
   \texttt{-c, --cluster}&\verb#[-]#\textit{url}&explicitly select or reject (-) a specific site\\
   \texttt{-i, --index}&\verb#[-]#\textit{url}&explicitly select or reject (-) a specific index server\\
   \texttt{-e, --jobdescrstring}&\textit{filename}&string describing the job to be submitted\\
   \texttt{-f, --jobdescrfile}&\textit{filename}&file describing the job to be submitted\\
   \texttt{-j, --joblist}&\textit{filename}&file where user's job information will be stored\\
%   \texttt{-D, --dryrun}&&add dryrun option to the job description\\
   \texttt{-x, --dumpdescription}&&do not submit -- dump transformed job description to stdout\\
%   \texttt{-U, --unknownattr}&&allow unknown attributes in the job description\\
   \texttt{-b, --broker}&\textit{string}&select broker method (default is Random)\\
%   \texttt{-n, --dolocalsandbox}& &store job descriptions in local sandbox (useful for eventual resubmission)\\
   \texttt{-t, --timeout}&\textit{seconds}&timeout in seconds (default 20)\\
   \texttt{-d, --debug}&\textit{debuglevel}&debug level, FATAL, ERROR, WARNING, INFO, VERBOSE or DEBUG - default WARNING\\
   \texttt{-z, --conffile}&\textit{filename}& configuration file (default {\$}HOME/.arc/client.conf)\\
   \texttt{-v, --version}&&print version information\\
   \texttt{-h, --help}&&print help page\\
   Arguments:&&\\
   \texttt{filename ...}&&file(s) describing the job(s) to be submitted\\
\end{longtable}

\begin{framed}
The \verb#-c# and \verb#-i# arguments accept meta-URLs of the format \texttt{GRID:URL}, where \texttt{GRID} indicates a Grid middleware flavour. Possible flavours are \texttt{ARC0}, \texttt{ARC1}, \texttt{CREAM} and \texttt{UNICORE}. For example, for index servers:
\begin{verbatim}
 ARC0:ldap://index.ng.org:2135/mds-vo-name=sweden,O=grid
 CREAM:ldap://cream.glite.org:2170/o=grid
\end{verbatim}
or clusters:

\verb# ARC0:ldap://ce.ng.eu:2135/nordugrid-cluster-name=ce.ng.eu,Mds-Vo-name=local,o=grid#

It is strongly recommended to use aliases for these long URLs. Aliases are specified in the configuration file (see Section~\ref{sec:client.conf}).
\end{framed}

%%% This might depend on the version of glibmm installed. In recent versions atleast it
%%% does not matter where the job description filename occurs on the command line. Any argument
%%% without a corresponding flag is taken to be a job description filename.
As a shorthand \texttt{-f} can be omitted if the job description file is put last on the commandline.

A simple \textit{"Hello World"} job can look like:

\begin{shaded}
 arcsub -c my-test-site job.jsdl
\end{shaded}

The \verb#-c# option can be repeated several times, for example:
\begin{verbatim}
    arcsub -c alias1 -c alias2 job.xrsl
\end{verbatim}
This will submit a job to either \verb#alias1# or \verb#alias2#. To submit a job to any site except
\verb#badsite#, use \verb#-# sign in front of the name:
\begin{verbatim}
    arcsub -c -badsite job.xrsl
\end{verbatim}

If option \verb#-c# is not given, the \verb#arcsub# command locates the available sites by querying the
Information System. Default index services for the Information System are specified in the
configuration template distributed with the middleware, and can be overwritten both in the user's
configuration (see Section~\ref{sec:client.conf}) and from the command line using option
\verb#-i#. Different Grids use different notation for such index services.

A user has to have valid credentials (see Section~\ref{sec:proxy}) and be authorised at the specified site. A test file \texttt{job.jsdl} is shown below.

\begin{lstlisting}[language=xml]
<?xml version="1.0" encoding="UTF-8"?>
<JobDefinition
 xmlns="http://schemas.ggf.org/jsdl/2005/11/jsdl"
 xmlns:posix="http://schemas.ggf.org/jsdl/2005/11/jsdl-posix">
 <JobDescription>
   <JobIdentification>
     <JobName>Hello World job</JobName>
   </JobIdentification>
   <Application>
     <posix:POSIXApplication>
       <posix:Executable>/bin/echo</posix:Executable>
       <posix:Argument>'Hello World'</posix:Argument>
       <posix:Output>out.txt</posix:Output>
       <posix:Error>err.txt</posix:Error>
     </posix:POSIXApplication>
   </Application>
 </JobDescription>
</JobDefinition>
\end{lstlisting}

\begin{framed}
   If a job is successfully submitted, a \textbf{job identifier}
   (\textit{job ID})\index{job ID} is printed to standard output.
\end{framed}

The job ID uniquely identifies the job while it is being executed. Job IDs
differ strongly between Grid flavours, but basically they have a form of a URL.
You should use Job ID as a handle to refer to the job when doing other
job manipulations, such as querying job status (\verb#arcstat#),
killing it (\verb#arckill#), re-submitting (\verb#arcresub#), or
retrieving the result (\verb#arcget#).

\begin{framed}
   Usually job ID is a valid URL for the job session directory. You can
   always use it to access the files related to the job, by using data
   management tools (see Chapter~\ref{sec:dm}).

   There may be exception for some computational services like CREAM 
   which do not support listing job session directory.
\end{framed}

\begin{framed}
   For advanced computational jobs which may need to communicate their 
   presence to some service outside there may be need for knowing own
   job ID. For jobs accepted by ARC computationsl services this information
   may be found in local (for job executable) environment variable 
   GRID\_GLOBAL\_JOBID. One need to take into account that this ID may be 
   different from one provided by arcsub. Example is ID provided by
   A-REX computational service. That service provides OGSA BES 
   compatible interface for job management and ID contains XML document
   according to OGSA BES specifications.
\end{framed}

The job description in one of the supported languages can be given, either as an
argument on the command line, or can be read from a file. Several jobs can be
requested at the same time by giving more than one filename argument, or by
repeating the \verb#-f# or \verb#-e# options. It is possible to mix \verb#-e#
and \verb#-f# options in the same \texttt{arcsub} command.

%%% The below path is OS dependend.
In order to keep track of submitted jobs, ARC client stores information in a
dedicated file, by default located in \texttt{{\$}HOME/.arc/jobs.xml}. It is
sometimes convenient to keep separate lists (e.g., for different kinds of jobs),
to be used later with e.g. \verb#arcstat#. This is achieved with the help of
\verb#-j# command line option.

The user interface transforms input job description into a format
that can be understood by the Grid services to which it is being
submitted. By specifying the \verb#--dumpdescription# option, such transformed
description is written to stdout instead of being submitted to the remote site.

% Description of brokers in a separate file
% text about brokers; may need to be extended and possibly re-used in another
% document, or moved as a separate subsection

Possible \index{broker}broker values for the \texttt{arcsub} command line option \verb#-b# are:
\begin{itemize}
 \item[--] \texttt{Random} -- ranks targets randomly (default)
 \item[--] \texttt{FastestQueue} -- ranks targets according to their queue length
 \item[--] \texttt{Benchmark[:name]} -- ranks targets according to a given benchmark, as specified by the \texttt{name}. If no benchmark is specified, CINT2000~\footnote{http://www.spec.org/cpu2000/CINT2000/} is used
 \item[--] \texttt{Data} -- ranks targets according the amount of megabytes of the
requested input files that are already in the computing resource’s cache.
 \item[--] \texttt{Python:$<$module$>$.$<$class$>$[:arguments]} -- ranks targets using any user-supplied custom Python broker module, optionally with broker arguments. Such module can reside anywhere in user's \texttt{PYTHONPATH}
% Below path is OS dependend.
 \item[--] \texttt{$<$otherbroker$>$[:arguments]} -- ranks targets using any user-supplied custom C++ broker plugin, optionally with broker arguments. Default location for broker plugins is \texttt{/usr/lib/arc} (may depend on the operating system), or the one specified by the \texttt{ARC\_PLUGIN\_PATH}.
\end{itemize}

% Below paths and description are OS dependend.
To write a custom broker in C++ one has to write a new specialization of the \texttt{Broker} base class and implement the \texttt{SortTargets} method in the new class. The class should be compiled as a loadable
module that has the proper ARC plugin descriptor for the new broker. For example, to build a broker plugin ``MyBroker'' one executes:
\begin{lstlisting}[language=sh]
  g++ -I /arc-install/include \
      -L /arc-install/lib \
      `pkg-config --cflags glibmm-2.4 libxml-2.0` \
      -o libaccmybroker.so -shared MyBroker.cpp
\end{lstlisting}
For more details, refer to \textit{libarclib} documentation~\cite{libarcclient}.



% By default local sandbox is enabled, and it is not possible atm to turn it
% off.
%If you plan to resubmit jobs, you will have to use command line option
%\verb#-n#, which will instruct the client to store complete job descriptions
%in a local sandbox, such that a resubmitted job will be identical to the
%original one.

It often happens that some sites that \verb#arcsub# has to contact
are slow to answer, or are down altogether. This will not prevent
you from submitting a job, but will slow down the submission. To
speed it up, you may want to specify a shorter timeout (default is
20 seconds) with the \verb#-t# option:
\begin{verbatim}
    arcsub -t 5 myjob.jsdl
\end{verbatim}

Default value for the timeout can be set in the user's configuration file.

If you would like to get diagnostics of the process of resource
discovery and requirements matching, a very useful option is
\verb#-d#. The following command:
\begin{verbatim}
    arcsub -d VERBOSE myjob.xrsl
\end{verbatim}

will print out the steps taken by the ARC client to find the
best cluster satisfying your job requirements. Possible diagnostics degrees, in the
order of increasing verbosity, are: \texttt{FATAL}, \texttt{ERROR}, \texttt{WARNING},
\texttt{INFO}, \texttt{VERBOSE} and \texttt{DEBUG}. Default is \texttt{WARNING}, and
it can be set to another value in the user's configuration file.

Default configuration file is \texttt{{\$}HOME/.arc/client.conf}. However, a user can choose any
other pre-defined configuration through option \verb#-z#.

Command line option \verb#-v# prints out version of the installed ARC client, and option
\verb#-h# provides a short help text.


\subsection{arcstat}
\label{sec:arcstat}

\begin{shaded}
   \uicommand{arcstat [options] [job ...]}
\end{shaded}
\versions{ARC 0.9}
\begin{longtable}{llp{8cm}}
   Options:&&\\
   \texttt{-a, --all}& & all jobs\\
   \texttt{-j, --joblist}& \textit{filename}& file containing a list of jobIDs\\
   \texttt{-c, --cluster}&\verb#[-]#\textit{name}&explicitly select or reject a specific site\\
%   \texttt{-c, --cluster}& & show information about a site (cluster)\\
   \texttt{-s, --status}& \textit{statusstr} &only select jobs whose status is \textit{statusstr}\\
%   \texttt{-i, --index}& \textit{url} &URL of an index service\\
   \texttt{-l, --long}& & long format (extended information)\\
   \texttt{-S, --sort}& \textit{criterion} & sort jobs according to jobid, submissiontime or jobname\\
   \texttt{-R, --rsort}& \textit{criterion} & reverse sorting of jobs according to jobid, submissiontime or jobname\\

   \texttt{-t, --timeout}& \textit{time}& timeout for queries (default 20 sec)\\
   \texttt{-d, --debug}& \textit{debuglevel}&debug level is one of  FATAL, ERROR, WARNING, INFO, VERBOSE or DEBUG\\
   \texttt{-z, --conffile}&\textit{filename}& configuration file (default {\$}HOME/.arc/client.conf)\\
   \texttt{-v, --version}& & print version information\\
   \texttt{-h, --help}& & print help page\\
   Arguments:&&\\
   \texttt{job ...} && list of job IDs and/or jobnames\\
\end{longtable}

The arcstat command returns the status of jobs in the Grid, and is typically issued with a
job ID (as returned by \verb#arcsub#) as an argument. It is also possible to use job name instead of
ID, but if several jobs have identical names, information will be collected about all of them.
More than one job ID and/or name can be given.

\begin{framed}
  When several of the \verb#-a#,  \verb#-j#, \verb#-c#, \verb#-s# and \verb#[job...]#
  command line options are specified, the comand returns information for \textbf{ALL} jobs that
  match either of the criteria (logical OR).

  For example, \verb#arcstat -s Finished -c mycluster <jobid># will return information about all finished
  jobs on the Grid, plus about all jobs (in any state) on the cluster \verb#mycluster#, plus about the job \verb#<jobid>#.
\end{framed}

If the \verb#-l# option is given, extended information is printed.

Jobs can be sorted according to the \texttt{jobid},
\texttt{submissiontime} or \texttt{jobname}, either in normal or
reverse order. By using the \verb#--sort# or \verb#--rsort# option
followed by the desired ordering (\texttt{jobid},
\texttt{submissiontime} or \texttt{jobname}), jobs will be sorted in
normal or reverse order. Note that the options \verb#--sort# and
\verb#--rsort# cannot be used at the same time.

Options \verb#-a#, \verb#-c#, \verb#-s# and \verb#-j# do not use job ID or names. By
specifying the \verb#-a# option, the status of all active jobs will be shown. If the \verb#-j# option
is used, the list of jobs is read from a file with the specified filename, instead of
the default one (\texttt{{\$}HOME/.arc/jobs.xml}).

Option \verb#-c# accepts arguments in the \texttt{GRID:URL} notation,
as explained in the description of \texttt{arcsub}, or their aliases as
specified in the configuration file.

Different sites may report different job states, depending on
the installed grid middleware version. Typical values can be e.g.
``Accepted'', ``Preparing'', ``Running'', ``Finished'' or ``Deleted''.
Please refer to the respective middleware documentation for job state model
description.

Command line option \verb#-s# will instruct the client to display information
of only those jobs which status matches the instruction. This option must be given
together with either \verb#-a# or \verb#-c# ones, e.g.:
\begin{verbatim}
    arcstat -as Finished
\end{verbatim}

Other command line options are identical to those of \verb#arcsub#.

\subsection{arccat}
\label{sec:arccat}

It is often useful to monitor the job progress by checking what it
prints on the standard output or error. The command \texttt{arccat}
\index{arccat}\index{commands:arccat} assists here, extracting the
corresponding information from the execution cluster and dumping it
on the user's screen. It works both for running tasks and for the
finished ones. This allows a user to check the output of the
finished task without actually retreiving it.

\hspace*{0.5cm}
\begin{shaded}
   \uicommand{arccat [options] [job ...]}
\end{shaded}
\versions{ARC 0.9}
\begin{longtable}{llp{8cm}}
   Options:&&\\
   \texttt{-a, --all}& & all jobs\\
   \texttt{-j, --joblist}& \textit{filename} & file containing a list of job IDs\\
   \texttt{-c, --cluster}&\verb#[-]#\textit{url}&explicitly select or reject (-) a specific site\\
   \texttt{-s, --status}& \textit{statusstr} &only select jobs whose status is \textit{statusstr}\\
   \texttt{-o, --stdout}& & show the stdout of the job (default)\\
   \texttt{-e, --stderr}& & show the stderr of the job\\
   \texttt{-l, --gmlog}& & show the grid manager's error log of the job\\
   \texttt{-t, --timeout}& \textit{time} & timeout for queries (default 20 sec)\\
   \texttt{-d, --debug}& \textit{debuglevel}&debug level is one of  FATAL, ERROR, WARNING, INFO, VERBOSE or DEBUG\\
   \texttt{-z, --conffile}&\textit{filename}& configuration file (default {\$}HOME/.arc/client.conf)\\
   \texttt{-v, --version}& & print version information\\
   \texttt{-h, --help}& & print help page\\
   Arguments:&&\\
   \texttt{job ...} && list of job IDs and/or jobnames\\
\end{longtable}

The \texttt{arccat} command returns the standard output of a job
(\texttt{-o} option), the standard error (\texttt{-e} option) or
errors reported by either Grid Manager or A-REX (\texttt{-l} option).

Other command line options have the same meaning as in \verb#arcstat#.

\begin{framed}
  When several of the \verb#-a#,  \verb#-j#, \verb#-c#, \verb#-s# and \verb#[job...]#
  command line options are specified, the comand prints logs for \textbf{ALL} jobs that
  match either of the criteria (logical OR).

  For example, \verb#arccat -s Finished -c mycluster <jobid># will print logs of all finished
  jobs on the Grid, plus of all jobs (in any state) on the cluster \verb#mycluster#, plus of the job \verb#<jobid>#.
\end{framed}


\subsection{arcget}
\label{sec:arcget}

To retrieve the results of a finished job, the \texttt{arcget}
\index{arcget}\index{commands:arcget} command should be used. It
will transfer the files specified for download in job description
to the user's computer.

\hspace*{0.5cm}
\begin{shaded}
   \uicommand{arcget [options] [job ...]}
\end{shaded}
\versions{ARC 0.9}
\begin{longtable}{llp{8cm}}
   Options:&&\\
   \texttt{-a, --all}& & all jobs\\
   \texttt{-j, --joblist}& \textit{filename} & file containing a list of jobIDs\\
   \texttt{-c, --cluster}&\verb#[-]#\textit{name}&explicitly select or reject a specific site (cluster)\\
   \texttt{-s, --status}& \textit{statusstr} &only select jobs whose status is \textit{statusstr}\\
   \texttt{-D, --dir} & \textit{dirname} & download path (the job directory will be created in that location)\\
   \texttt{-k, --keep}& & keep files in the Grid (do not clean)\\
   \texttt{-t, --timeout}& \textit{time} & timeout for queries (default 20 sec)\\
   \texttt{-d, --debug}& \textit{debuglevel}&debug level is one of  FATAL, ERROR, WARNING, INFO, VERBOSE or DEBUG\\
   \texttt{-z, --conffile}&\textit{filename}& configuration file (default {\$}HOME/.arc/client.conf)\\
   \texttt{-v, --version}& & print version information\\
   \texttt{-h, --help}& & print help page\\
   Arguments:&&\\
   \texttt{job ...} && list of job IDs and/or jobnames\\
\end{longtable}

Only the results of jobs that have finished can be downloaded. Just like in \verb#arcstat#
and \verb#arccat# cases, the job can be referred to either by the \texttt{jobID} that was returned by
\texttt{arcsub} at submission time, or by its name, if the job
description contained a job name attribute.

By default, the job is downloaded into a newly created directory in the current path, with the
name typically being a large random number. In order to instruct \verb#arcget# to use another
path, use option \verb#-D# (note the capital ``D''), e.g.
\begin{verbatim}
    arcget -D /tmp/myjobs "Test job nr 1"
\end{verbatim}

\begin{framed}
 After downloading, your jobs will be erased from the execution site! Use command line option \verb#-k#
 to keep finished jobs in the Grid.
\end{framed}

Other command line options are identical to those of e.g. \verb#arcstat#.

\begin{framed}
  When several of the \verb#-a#,  \verb#-j#, \verb#-c#, \verb#-s# and \verb#[job...]#
  command line options are specified, the comand retrieves \textbf{ALL} jobs that
  match either of the criteria (logical OR).

  For example, \verb#arcget -s Finished -c mycluster <jobid># will retrieve all finished
  jobs on the Grid, plus all jobs (in any state) on the cluster \verb#mycluster#, plus the job \verb#<jobid>#.
\end{framed}


\subsection{arcsync}
\label{sec:arcsync}

It is advised to start every grid session by running \texttt{arcsync},
especially when changing workstations. The reason is that your job submission
history is cached on your machine, and if you are using ARC client
installations on different machines, your local lists of submitted jobs will be different. To synchronise
these lists with the information in the Information System, use the
\texttt{arcsync} \index{arcsync}\index{commands:arcsync} command.

\hspace*{0.5cm}
\begin{shaded}
   \uicommand{arcsync [options]}
\end{shaded}
\versions{ARC 0.9}
\begin{longtable}{llp{8cm}}
   Options:&&\\
   \texttt{-c, --cluster}&\verb#[-]#\textit{name}&explicitly select or reject a specific site\\
   \texttt{-i, --index}&\textit{url}&explicitly select or reject (-) a specific index server\\
   \texttt{-j, --joblist}&\textit{filename}&file where user's job information will be stored\\
   \texttt{-f, --force}& &don't ask for confirmation\\
%   \texttt{-m, --merge}& &merge the found jobs with the jobs already in the joblist\\
   \texttt{-T, --truncate}& &truncate the job list before synchronising\\
   \texttt{-t, --timeout}&\textit{seconds}&timeout in seconds (default 20)\\
   \texttt{-d, --debug}&\textit{debuglevel}&debug level, FATAL, ERROR, WARNING, INFO, VERBOSE or DEBUG - default WARNING\\
   \texttt{-z, --conffile}&\textit{filename}& configuration file (default {\$}HOME/.arc/client.conf)\\
   \texttt{-v, --version}&&print version information\\
   \texttt{-h, --help}&&print help page\\
\end{longtable}

The ARC client keeps a local list of jobs in the user's home
directory. If this file is lost,
corrupt, or the user wants to recreate the file on a different
workstation, the \texttt{arcsync} command will recreate this file from
the information available in the Information System.

Since  the  information  about  a job retrieved from a cluster can be slightly out of date if the user very recently
submitted or removed a job, a warning is issued when this command is run. The \verb#-f# option disables this warning.

If the job list is not empty when invoking syncronisation, the old jobs will be merged with the new jobs, unless
the \verb#-T# option is given (note the capital ``T''), in which case the job list will first be truncated and then the new jobs will be added.

\subsection{arcinfo}
\label{sec:arcinfo}

The \texttt{arcinfo} \index{arcinfo}\index{commands:arcinfo} command is used to obtain status information about clusters on the Grid.

\hspace*{0.5cm}
\begin{shaded}
   \uicommand{arcinfo [options]}
\end{shaded}
\versions{ARC 0.9}
\begin{longtable}{llp{8cm}}
   Options:&&\\
   \texttt{-c, --cluster}&\verb#[-]#\textit{name}&explicitly select or reject a specific site\\
   \texttt{-i, --index}&\textit{url}&explicitly select or reject (-) a specific index server\\
   \texttt{-l, --long}& & long format (extended information)\\
   \texttt{-t, --timeout}&\textit{seconds}&timeout in seconds (default 20)\\
   \texttt{-d, --debug}&\textit{debuglevel}&debug level, FATAL, ERROR, WARNING, INFO, VERBOSE or DEBUG - default WARNING\\
   \texttt{-z, --conffile}&\textit{filename}& configuration file (default {\$}HOME/.arc/client.conf)\\
   \texttt{-v, --version}&&print version information\\
   \texttt{-h, --help}&&print help page\\
\end{longtable}

The  \texttt{arcinfo}  command is used to obtain information about clusters and queues (\textit{targets})
available on the Grid. Either the \verb#--cluster# or \verb#--index# flag should be used to specify the
target(s) which should be queried for information. Both of these flags take a service endpoint as argument.
See \texttt{arcsub} and the configuration notes in Section~\ref{sec:client.conf} for description of these.

Detailed information about queried computing services can be obtained by specifying the \verb#--long# flag.

When  specifying  the  \verb#--index#  flag,  the  information about the computing services registered at the index server will
be queried, rather than the status of the index server itself.


\subsection{arckill}
\label{sec:arckill}

It happens that a user may wish to cancel a job. This is done by using
the \texttt{arckill} \index{arckill}\index{commands:arckill} command. A
job can be killed almost at any stage of processing through the Grid.

\hspace*{0.5cm}
\begin{shaded}
   \uicommand{arckill [options] [job ...]}
\end{shaded}
\versions{ARC 0.9}
\begin{longtable}{llp{8cm}}
   Options:&&\\
   \texttt{-a, --all}& & all jobs\\
   \texttt{-j, --joblist}& \textit{filename} & file containing a list of jobIDs\\
   \texttt{-c, --cluster}&\verb#[-]#\textit{url}&explicitly select or reject (-) a specific site\\
   \texttt{-s, --status}& \textit{statusstr} &only select jobs whose status is \textit{statusstr}\\
   \texttt{-k, --keep}& & keep files in the Grid (do not clean)\\
   \texttt{-t, --timeout}& \textit{time} & timeout for queries (default 20 sec)\\
   \texttt{-d, --debug}& \textit{debuglevel}&debug level is one of  FATAL, ERROR, WARNING, INFO, VERBOSE or DEBUG\\
   \texttt{-z, --conffile}&\textit{filename}& configuration file (default {\$}HOME/.arc/client.conf)\\
   \texttt{-v, --version}& & print version information\\
   \texttt{-h, --help}& & print help page\\
   Arguments:&&\\
   \texttt{job ...} && list of job IDs and/or jobnames\\
\end{longtable}

If a job is killed, its traces are being cleaned from the Grid. If you wish to keep the killed job
in the system, e.g. for a post-mortem analysis, use the \verb#-k# option.

\begin{framed}
   Job cancellation is an asynchronous process, such that it
   may take a few minutes before the job is actually cancelled.
\end{framed}

Command line options have the same meaning as the corresponding ones of \verb#arcstat# and others.

\begin{framed}
  When several of the \verb#-a#,  \verb#-j#, \verb#-c#, \verb#-s# and \verb#[job...]#
  command line options are specified, the comand kills \textbf{ALL} jobs that
  match either of the criteria (logical OR).

  For example, \verb#arckill -s INLRMS:R -c mycluster <jobid># will kill all running
  jobs on the Grid, plus all jobs (in any state) on the cluster \verb#mycluster#, plus the job \verb#<jobid>#.
\end{framed}


\subsection{arcclean}
\label{sec:arcclean}

If a job fails or gets killed with \verb#-k# option, or when you are not willing
to retrieve the results for some reasons, a good practice for users is not to wait
for the system to clean up the job leftovers, but to use
\texttt{arcclean}\index{arcclean}\index{commands:arcclean} to release the disk
space and to remove the job ID from the list of submitted jobs and from the Information System.

\hspace*{0.5cm}
\begin{shaded}
   \uicommand{arcclean [options] [job ...]}
\end{shaded}
\versions{ARC 0.9}
\begin{longtable}{llp{8cm}}
   Options:&&\\
   \texttt{-a, --all}& & all jobs\\
   \texttt{-j, --joblist}& \textit{filename} & file containing a list of jobIDs\\
   \texttt{-c, --cluster}&\verb#[-]#\textit{name}&explicitly select or reject a specific site (cluster)\\
   \texttt{-s, --status}& \textit{statusstr} &only select jobs whose status is \textit{statusstr}\\
   \texttt{-f, --force} & & removes the job ID from the local list even if the job is not found on the Grid\\
   \texttt{-t, --timeout}& \textit{time} & timeout for queries (default 20 sec)\\
   \texttt{-d, --debug}& \textit{debuglevel}&debug level is one of  FATAL, ERROR, WARNING, INFO, VERBOSE or DEBUG\\
   \texttt{-z, --conffile}&\textit{filename}& configuration file (default {\$}HOME/.arc/client.conf)\\
   \texttt{-v, --version}& & print version information\\
   \texttt{-h, --help}& & print help page\\
   Arguments:&&\\
   \texttt{job ...} && list of job IDs and/or jobnames\\
\end{longtable}

Only jobs that have finished or were cancelled can be cleaned.

It happens ever so often that the job is cleaned by the system, or is otherwise unreachable, and yet your
local job list file still has it listed. Use \verb#-f# option in this case to forcefully remove such stale job
information from the local list.

Other command line options have the same meaning as the corresponding ones of \verb#arcstat# and others.

\begin{framed}
  When several of the \verb#-a#,  \verb#-j#, \verb#-c#, \verb#-s# and \verb#[job...]#
  command line options are specified, the comand cleans \textbf{ALL} jobs that
  match either of the criteria (logical OR).

  For example, \verb#arcclean -s FAILED -c mycluster <jobid># will clean all failed
  jobs on the Grid, plus all jobs (in any state) on the cluster \verb#mycluster#, plus the job \verb#<jobid>#.
\end{framed}


\subsection{arcrenew}
\label{sec:arcrenew}

Quite often, the user proxy expires while the job is still running (or
waiting in a queue). In case such job has to upload output files to a
Grid location (Storage Element), it will fail. By using the \texttt{arcrenew}
\index{arcrenew}\index{commands:arcrenew} command, users can upload
a new proxy to the job. This can be done while a job is still running,
thus preventing it from failing

If a job has failed in file upload due to expired proxy, \texttt{arcrenew}
can be issued whithin 24 hours (or whatever is
the expiration time set by the site) after the job
end, which must be followed by \texttt{arcresume}. The Grid
Manager or A-REX will then attempt to finalize
the job by uploading the output files to the desired location.

\hspace*{0.5cm}
\begin{shaded}
   \uicommand{arcrenew [options] [job ...]}
\end{shaded}
\versions{ARC 0.9}
\begin{longtable}{llp{8cm}}
   Options:&&\\
   \texttt{-a, --all}& & all jobs\\
   \texttt{-j, --joblist}& \textit{filename} & file containing a list of jobIDs\\
   \texttt{-c, --cluster}&\verb#[-]#\textit{name}&explicitly select or reject a specific site (cluster)\\
   \texttt{-s, --status}& \textit{statusstr} &only select jobs whose status is \textit{statusstr}\\
   \texttt{-t, --timeout}& \textit{time} & timeout for queries (default 20 sec)\\
   \texttt{-d, --debug}& \textit{debuglevel}&debug level is one of  FATAL, ERROR, WARNING, INFO, VERBOSE or DEBUG\\
   \texttt{-z, --conffile}&\textit{filename}& configuration file (default {\$}HOME/.arc/client.conf)\\
   \texttt{-v, --version}& & print version information\\
   \texttt{-h, --help}& & print help page\\
   Arguments:&&\\
   \texttt{job ...} && list of job IDs and/or jobnames\\
\end{longtable}

\begin{framed}
 Prior to using \texttt{arcrenew}, be sure to actually create the
new proxy by running \verb#arcproxy#!
\end{framed}

Command line options have the same meaning as the corresponding ones of \verb#arcstat# and others.

\begin{framed}
  When several of the \verb#-a#,  \verb#-j#, \verb#-c#, \verb#-s# and \verb#[job...]#
  command line options are specified, the comand renews proxies for \textbf{ALL} jobs that
  match either of the criteria (logical OR).

  For example, \verb#arcrenew -s FAILED -c mycluster <jobid># will renew proxies of all failed
  jobs on the Grid, plus of all jobs (in any state) on the cluster \verb#mycluster#, plus of the job \verb#<jobid>#.
\end{framed}


\subsection{arcresume}
\label{sec:arcresume}

In some cases a user may want to restart a failed job, for example, when input
files become available, or the storage element for the output files came back
online, or when a proxy is renewed with \texttt{arcrenew}. This can be done using
the \texttt{arcresume}\index{arcresume}\index{commands:arcresume} command.

\begin{framed}
Make sure your proxy is still valid, or when uncertain, run \verb#arcproxy# followed by
\verb#arcrenew# before \verb#arcresume#. The job will be resumed from the state where it has failed.
\end{framed}

\hspace*{0.5cm}
\begin{shaded}
   \uicommand{arcresume [options] [job ...]}
\end{shaded}
\versions{ARC 0.9}
\begin{longtable}{llp{8cm}}
   \texttt{-a, --all}& & all jobs\\
   \texttt{-j, --joblist}& \textit{filename} & file containing a list of jobIDs\\
   \texttt{-c, --cluster}&\verb#[-]#\textit{name}&explicitly select or reject a specific site (cluster)\\
   \texttt{-s, --status}& \textit{statusstr} &only select jobs whose status is \textit{statusstr}\\
   \texttt{-t, --timeout}& \textit{time} & timeout for queries (default 20 sec)\\
   \texttt{-d, --debug}& \textit{debuglevel}&debug level is one of  FATAL, ERROR, WARNING, INFO, VERBOSE or DEBUG\\
   \texttt{-z, --conffile}&\textit{filename}& configuration file (default {\$}HOME/.arc/client.conf)\\
   \texttt{-v, --version}& & print version information\\
   \texttt{-h, --help}& & print help page\\
   Arguments:&&\\
   \texttt{job ...} && list of job IDs and/or jobnames\\
\end{longtable}

Command line options have the same meaning as the corresponding ones of \verb#arcstat# and others.

\begin{framed}
  When several of the \verb#-a#,  \verb#-j#, \verb#-c#, \verb#-s# and \verb#[job...]#
  command line options are specified, the comand resumes \textbf{ALL} jobs that
  match either of the criteria (logical OR).

  For example, \verb#arcresume -s FAILED -c mycluster <jobid># will resume all failed
  jobs on the Grid, plus all jobs (in any state) on the cluster \verb#mycluster#, plus the job \verb#<jobid>#.
\end{framed}


\subsection{arcresub}
\label{sec:arcresub}

Quite often it happens that a user would like to re-submit a job, but
has difficulties recovering the original job description xRSL file.
This happens when xRSL files are created by scripts on-fly, and
matching of xRSL to the job ID is not straightforward. The utility
called \texttt{arcresub}\index{arcresub}\index{commands:arcresub} helps
in such situations, allowing users to resubmit jobs.

\hspace*{0.5cm}
\begin{shaded}
   \uicommand{arcresub [options] [job ...]}
\end{shaded}
\versions{ARC 0.9}
\begin{longtable}{llp{8cm}}
   Options:&&\\
   \texttt{-a, --all}& & all jobs\\
   \texttt{-i, --index}&\textit{url}&explicitly select or reject (-) a specific index server\\
   \texttt{-j, --joblist}& \textit{filename} & file containing a list of jobIDs\\
   \texttt{-c, --cluster}&\verb#[-]#\textit{name}&explicitly select or reject a specific source site\\
   \texttt{-q, --qluster}&\verb#[-]#\textit{name}&explicitly select or reject a specific site as re-submission target\\
   \texttt{-m, --same}& &re-submit to the same site\\
   \texttt{-s, --status}& \textit{statusstr} &only select jobs whose status is \textit{statusstr}\\
   \texttt{-x, --dumpdescription}&&do not submit -- dump transformed job description to stdout\\
   \texttt{-k, --keep}& & keep files in the Grid (do not clean)\\
   \texttt{-b, --broker}&\textit{string}&select broker method (default is Random)\\
   \texttt{-t, --timeout}& \textit{time} & timeout for queries (default 20 sec)\\
   \texttt{-d, --debug}& \textit{debuglevel}&debug level is one of  FATAL, ERROR, WARNING, INFO, VERBOSE or DEBUG\\
   \texttt{-z, --conffile}&\textit{filename}& configuration file (default {\$}HOME/.arc/client.conf)\\
   \texttt{-v, --version}& & print version information\\
   \texttt{-h, --help}& & print help page\\
   Arguments:&&\\
   \texttt{job ...} && list of job IDs and/or jobnames\\
\end{longtable}

\begin{framed}
   Only jobs where the \verb#gmlog#\index{gmlog} attribute was
   specified in the job description can be resubmitted.
\end{framed}

More  than  one  jobid and/or jobname can be given. If several
jobs were submitted with the same jobname all those jobs will be resubmitted.

Upon resubmission of a job the corresponding job description will be
fetched from the local job list file. If input files have changed since the
original job submission, the job no longer remains the same job
and will therefore not be resubmitted. To make sure the job is always resubmittable,
submit it with \verb#arcsub -n#.

In case the job description is not found in the joblist, an attempt will be made to
retrieve it from the cluster holding the orignal job.  This however
may fail, since both the submission client and
the cluster can have made modifications to the job description.

Upon resubmision the job will receive a new job ID. The old job ID will be kept
in the local job list file, enabling future back tracing of the resubmitted job.

Regarding command line options, \verb#arcresub# behaves much like \verb#arcsub#, except that
\verb#-c# in this case indicates not the submission target site, but on the contrary, the \textbf{site
from which the jobs will be resubmitted}. Submission target site is specified with
option \verb#-q#. If you wish to re-submit each job to the same site, use option \verb#-m#.

If the original job was successfully killed, its traces will be removed from the execution site,
unless the \verb#-k# option is specified.

\begin{framed}
  When several of the \verb#-a#,  \verb#-j#, \verb#-c#, \verb#-s# and \verb#[job...]#
  command line options are specified, the comand resubmits \textbf{ALL} jobs that
  match either of the criteria (logical OR).

  For example, \verb#arcresub -s FAILED -c mycluster <jobid># will resubmit all failed
  jobs on the Grid, plus all jobs (in any state) on the cluster \verb#mycluster#, plus the job \verb#<jobid>#.
\end{framed}


\subsection{arcmigrate}
\label{sec:arcmigrate}

Quite often jobs end up stuck in long queues, and users wish to migrate them to a better
resource. Command \texttt{arcmigrate}\index{arcmigrate}\index{commands:arcmigrate} is
triggering this migration. It applies only to jobs submitted to A-REX, as other Grid
execution services do not support this functionality.

\hspace*{0.5cm}
\begin{shaded}
   \uicommand{arcmigrate [options] [job ...]}
\end{shaded}
\versions{ARC 0.9}
\begin{longtable}{llp{8cm}}
   Options:&&\\
   \texttt{-a, --all}& & all jobs\\
   \texttt{-i, --index}&\textit{url}&explicitly select or reject (-) a specific index server\\
   \texttt{-j, --joblist}& \textit{filename} & file containing a list of jobIDs\\
   \texttt{-c, --cluster}&\verb#[-]#\textit{name}&explicitly select or reject a specific site (cluster)\\
   \texttt{-q, --qluster}&\verb#[-]#\textit{name}&explicitly select or reject a specific site as re-submission target\\
   \texttt{-f, --forcemigration}& & force migration, ignoring kill failure\\
   \texttt{-b, --broker}&\textit{string}&select broker method (default is Random)\\
   \texttt{-t, --timeout}& \textit{time} & timeout for queries (default 20 sec)\\
   \texttt{-d, --debug}& \textit{debuglevel}&debug level is one of  FATAL, ERROR, WARNING, INFO, VERBOSE or DEBUG\\
   \texttt{-z, --conffile}&\textit{filename}& configuration file (default {\$}HOME/.arc/client.conf)\\
   \texttt{-v, --version}& & print version information\\
   \texttt{-h, --help}& & print help page\\
   Arguments:&&\\
   \texttt{job ...} && list of job IDs and/or jobnames\\
\end{longtable}

\begin{framed}
Currently only jobs having the A-REX status ``Running'', ``Executing'' or ``Queuing'' can be migrated
\end{framed}

Command line options \verb#-c# and \verb #-q# are interpreted in the same way as in \verb#arcresub#, namely,
\verb#-c# indicates ``from'' and \verb #-q# -- ``to'' which site the job will be migrated.

If the job(s) is successfully migrated, a new job ID(s) is printed out. This jobID uniquely
identifies the job while it is being executed.

\begin{framed}
  When several of the \verb#-a#,  \verb#-j#, \verb#-c#, \verb#-s# and \verb#[job...]#
  command line options are specified, the comand migrates \textbf{ALL} jobs that
  match either of the criteria (logical OR).

  For example, \verb#arcmigrate -s FAILED -c mycluster <jobid># will migrate all failed
  jobs on the Grid, plus all jobs (in any state) on the cluster \verb#mycluster#, plus the job \verb#<jobid>#.
\end{framed}
