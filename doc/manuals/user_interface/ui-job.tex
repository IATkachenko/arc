\subsection{arcsub}\label{sec:arcsub}
The \texttt{arcsub}\index{arcsub}\index{commands:arcsub} command is the
most essential one, as it is used for submitting jobs to the Grid
resources. \index{submit job}. \texttt{arcsub} matches user's job
description to the information collected from the Grid, and the
optimal site is being selected for job submission. The job description
is then being forwarded to that site, in order to be submitted to the
Local Resource Management System (LRMS), which can be, e.g., PBS or
Condor or SGE etc.

\hspace*{0.5cm}
\begin{shaded}
   \uicommand{arcsub [options]}%$<$task ...$>$
\end{shaded}
\versions{ARC 0.9}
\begin{longtable}{llp{8cm}}
   Options:&&\\
   \texttt{-c, --cluster}&\verb#[-]#\textit{url}&explicitly select or reject a specific site (cluster)\\
   \texttt{-i, --indexurl}&\textit{url}&URL of an index server\\
   \texttt{-e, --jobdescrstring}&\textit{filename}&string describing the job to be submitted\\
   \texttt{-f, --jobdescrfile}&\textit{filename}&file describing the job to be submitted\\
   \texttt{-j, --joblist}&\textit{filename}&file where the job IDs will be stored\\
   \texttt{-D, --dryrun}&&add dryrun option to the job description\\
   \texttt{-x, --dumpdescription}&&do not submit -- dump transformed job description to stdout\\
   \texttt{-U, --unknownattr}&&allow unknown attributes in the job description\\
   \texttt{-t, --timeout}&\textit{time}&timeout for queries in seconds (default 20)\\
   \texttt{-d, --debug}&\textit{debuglevel}&debug level, FATAL, ERROR, WARNING, INFO, DEBUG or VERBOSE - default WARNING\\
   \texttt{-z, --conffile}&\textit{filename}& configuration file (default {\$}HOME/.arc/client.xml)\\
   \texttt{-v, --version}&&print version information\\
   \texttt{-h, --help}&&print help page\\
%   Arguments:&&\\
%   \texttt{ task ...}&&strings or files describing the jobs to be submitted\\
\end{longtable}



\begin{framed}
The -c and -i arguments accept ``urls'' of the format GRID:URL,

e.g. for index servers:

ARC0:ldap://grid.tsl.uu.se:2135/mds-vo-name=sweden,O=grid

CREAM:ldap://cream.grid.upjs.sk:2170/o=grid

or clusters:

ARC0:ldap://grid.tsl.uu.se:2135/nordugrid-cluster-name=grid.tsl.uu.se,Mds-Vo-name=local,o=grid
\end{framed}

As a shorthand \texttt{-f} can be omitted if the job description file is put last on the commandline.

A simple \textit{"Hello World"} job could look like:

\begin{shaded}
\uicommand{arcsub -c ARC0:ldap://grid.tsl.uu.se:2135/nordugrid-cluster-name=grid.tsl.uu.se,Mds-Vo-name=local,o=grid -f job.jsdl}
\end{shaded}

Assuming that the url represents an ARC0 cluster that the user is authorized to submit to and job.jsdl is the file below.

\begin{framed}
\begin{verbatim}
<?xml version="1.0" encoding="UTF-8"?>
<JobDefinition
 xmlns="http://schemas.ggf.org/jsdl/2005/11/jsdl"
 xmlns:posix="http://schemas.ggf.org/jsdl/2005/11/jsdl-posix">
 <JobDescription>
   <JobIdentification>
     <JobName>Hello World job</JobName>
   </JobIdentification>
   <Application>
     <posix:POSIXApplication>
       <posix:Executable>/bin/echo</posix:Executable>
       <posix:Argument>'Hello World'</posix:Argument>
       <posix:Output>out.txt</posix:Output>
       <posix:Error>err.txt</posix:Error>
     </posix:POSIXApplication>
   </Application>
   <DataStaging>
     <FileName>out.txt</FileName>
     <CreationFlag>overwrite</CreationFlag>
     <DeleteOnTermination>false</DeleteOnTermination>
   </DataStaging>
   <DataStaging>
     <FileName>err.txt</FileName>
     <CreationFlag>overwrite</CreationFlag>
     <DeleteOnTermination>false</DeleteOnTermination>
   </DataStaging>
 </JobDescription>
</JobDefinition>
\end{verbatim}
\end{framed}

\subsection{arcstat}
\label{sec:arcstat}


\begin{shaded}
   \uicommand{arcstat [options] [job ...]}
\end{shaded}
\versions{ARC 0.9}
\begin{longtable}{llp{8cm}}
   Options:&&\\
   \texttt{-a, --all}& & all jobs\\
   \texttt{-i, --joblist}& \textit{filename}& file containing a list of jobIDs\\
   \texttt{-c, --cluster}& & show information about a site (cluster)\\
   \texttt{-s, --status}& \textit{statusstr} &only select jobs whose status is \textit{statusstr}\\
   \texttt{-i, --indexurl}& \textit{url} &URL of an index service\\
   \texttt{-q, --queues}&&show information about clusters and queues\\
   \texttt{-l, --long}& & long format (extended information)\\
   \texttt{-t, --timeout}& \textit{time}& timeout for queries (default 20 sec)\\
   \texttt{-d, --debug}& \textit{debuglevel}&debug level is one of  FATAL, ERROR, WARNING, INFO, DEBUG or VERBOSE\\
   \texttt{-z, --conffile}&\textit{filename}& configuration file (default {\$}HOME/.arc/client.xml)\\
   \texttt{-v, --version}& & print version information\\
   \texttt{-h, --help}& & print help page\\
   Arguments:&&\\
   \texttt{job ...} && list of job IDs and/or jobnames\\
\end{longtable}

The arcstat command returns the status of jobs submitted to the Grid. Then \texttt{-c} and \texttt{-i} accept arguments in the GRID:URL notation explained in the description of \texttt{arcsub}.

Different sites may report slightly different job states, depending on
the installed software version.% A summary of essential job
%states\index{job:states} is:

%  ---  Does the table below need updating? ---

% \begin{longtable}{l|l|p{8cm}}
% ARC 0.3, ARC 0.4 & ARC 0.5, ARC0.6 & Description\\ \hline
%                    &\texttt{ACCEPTING} &job has reached the site\\
% \texttt{ACCEPTED}  &\texttt{ACCEPTED}  &job submitted but not yet processed\\
% \texttt{PREPARING} &\texttt{PREPARING} &input files are being retreived\\
%                    &\texttt{PREPARED}  &input files are retreived\\
% \texttt{SUBMITTING}&\texttt{SUBMITTING}&interaction with LRMS ongoing\\
% \texttt{INLRMS: Q} &\texttt{INLRMS:Q}  &job is queued by LRMS\\
% \texttt{INLRMS: R} &\texttt{INLRMS:R}  &job is running\\
%                    &\texttt{INLRMS:S}  &job is suspended\\
%                    &\texttt{INLRMS:E}  &job is finishing in LRMS\\
%                    &\texttt{INLRMS:O}  &job is in any other LRMS state\\
% \texttt{CANCELING} &\texttt{KILLING}   &job is being cancelled by user request\\
%                    &\texttt{EXECUTED}  &job is completed in LRMS\\
% \texttt{FINISHING} &\texttt{FINISHING} &output files are being transferred\\
% \texttt{FINISHED}  &\texttt{FINISHED}  &job is finished\\
%                    &\texttt{FAILED}    &job is finished with an error\\
%                    &\texttt{KILLED}    &job is cancelled by user request\\
% \texttt{DELETED}   &\texttt{DELETED}   &job is removed due to expiration time\\
% \end{longtable}

\subsection{arccat}
\label{sec:arccat}

It is often useful to monitor the job progress by checking what it
prints on the standard output or error. The command \texttt{arccat}
\index{arccat}\index{commands:arccat} assists here, extracting the
corresponding information from the execution cluster and pasting it
on the user's screen. It works both for running tasks and for the
finished ones. This allows a user to check the output of the
finished task without actually retreiving it.

\hspace*{0.5cm}
\begin{shaded}
   \uicommand{arccat [options] [job ...]}
\end{shaded}
\versions{ARC 0.9}
\begin{longtable}{llp{8cm}}
   Options:&&\\
   \texttt{-a, --all}& & all jobs\\
   \texttt{-i, --joblist}& \textit{filename} & file containing a list of job IDs\\
   \texttt{-c, --cluster}& & show information about clusters\\
   \texttt{-s, --status}& \textit{statusstr} &only select jobs whose status is \textit{statusstr}\\
   \texttt{-o, --stdout}& & show the stdout of the job (default)\\
   \texttt{-e, --stderr}& & show the stderr of the job\\
   \texttt{-l, --gmlog}& & show the grid manager's error log of the job\\
   \texttt{-t, --timeout}& \textit{time} & timeout for queries (default 20 sec)\\
   \texttt{-d, --debug}& \textit{debuglevel}&debug level is one of  FATAL, ERROR, WARNING, INFO, DEBUG or VERBOSE\\
   \texttt{-z, --conffile}&\textit{filename}& configuration file (default {\$}HOME/.arc/client.xml)\\
   \texttt{-v, --version}& & print version information\\
   \texttt{-h, --help}& & print help page\\
   Arguments:&&\\
   \texttt{job ...} && list of job IDs and/or jobnames\\
\end{longtable}

The \texttt{arccat} command can return the standard output of a job
(\texttt{-o} option), the standard error (\texttt{-e} option) and the
errors reported by the Grid Manager (\texttt{-l} option).

\begin{framed}

\end{framed}


\subsection{arcget}
\label{sec:arcget}

To retrieve the results of a finished job, the \texttt{arcget}
\index{arcget}\index{commands:arcget} command should be used. It
will download the files specified by the \texttt{outputfiles}
attribute of job description to the user's computer.

\hspace*{0.5cm}
\begin{shaded}
   \uicommand{arcget [options] [job ...]}
\end{shaded}
\versions{ARC 0.9}
\begin{longtable}{llp{8cm}}
   Options:&&\\
   \texttt{-a, --all}& & all jobs\\
   \texttt{-i, --joblist}& \textit{filename} & file containing a list of jobIDs\\
   \texttt{-c, --cluster}&\verb#[-]#textem{name}&explicitly select or reject a specific site (cluster)\\
   \texttt{-s, --status}& \textit{statusstr} &only select jobs whose status is \textit{statusstr}\\
   \texttt{-D, --dir} & \textit{dirname} & download directory (the job directory will be created in this directory)\\
   \texttt{-k, --keep}& & keep files on gatekeeper (do not clean)\\
   \texttt{-t, --timeout}& \textit{time} & timeout for queries (default 20 sec)\\
   \texttt{-d, --debug}& \textit{debuglevel}&debug level is one of  FATAL, ERROR, WARNING, INFO, DEBUG or VERBOSE\\
   \texttt{-z, --conffile}&\textit{filename}& configuration file (default {\$}HOME/.arc/client.xml)\\
   \texttt{-v, --version}& & print version information\\
   \texttt{-h, --help}& & print help page\\
   Arguments:&&\\
   \texttt{job ...} && list of job IDs and/or jobnames\\
\end{longtable}

Only the results of jobs that have finished can be downloaded. The job
can be referred to either by the \texttt{jobID} that was returned by
\texttt{arcsub} at submission time, or by its name, if the job
description contained a job name attribute.

\subsection{arckill}
\label{sec:arckill}

It happens that a user may wish to cancel a job. This is done by using
the \texttt{arckill} \index{arckill}\index{commands:arckill} command. A
job can be killed amost on any stage of processing through the Grid.

\hspace*{0.5cm}
\begin{shaded}
   \uicommand{arckill [options] [job ...]}
\end{shaded}
\versions{ARC 0.9}
\begin{longtable}{llp{8cm}}
   Options:&&\\
   \texttt{-a, --all}& & all jobs\\
   \texttt{-j, --joblist}& \textit{filename} & file containing a list of jobIDs\\
   \texttt{-c, --cluster}& & show information about clusters\\
   \texttt{-s, --status}& \textit{statusstr} &only select jobs whose status is \textit{statusstr}\\
   \texttt{-k, --keep}& & keep files on gatekeeper (do not clean)\\
   \texttt{-t, --timeout}& \textit{time} & timeout for queries (default 20 sec)\\
   \texttt{-d, --debug}& \textit{debuglevel}&debug level is one of  FATAL, ERROR, WARNING, INFO, DEBUG or VERBOSE\\
   \texttt{-z, --conffile}&\textit{filename}& configuration file (default {\$}HOME/.arc/client.xml)\\
   \texttt{-v, --version}& & print version information\\
   \texttt{-h, --help}& & print help page\\
   Arguments:&&\\
   \texttt{job ...} && list of job IDs and/or jobnames\\
\end{longtable}

\begin{framed}
   Job cancellation is an asynchronous process, such that it
   may take a few minutes before the job is actually cancelled.
\end{framed}

% \subsection{arcresub}
% \label{sec:arcresub}
% 
% Quite often it happens that a user would like to re-submit a job, but
% has difficulties recovering the original job description xRSL file.
% This happens when xRSL files are created by scripts on-fly, and
% matching of xRSL to the job ID is not straightforward. The utility
% called \texttt{arcresub}\index{arcresub}\index{commands:arcresub} helps
% in such situations, allowing users to resubmit jobs known only by
% their job IDs.
% 
% \begin{framed}
%    Only jobs where the \verb#gmlog#\index{gmlog} attribute was
%    specified in the job description can be resubmitted.
% \end{framed}
% 
% \hspace*{0.5cm}
% \begin{shaded}
%    \uicommand{arcresub [options] [job ...]}
% \end{shaded}
% \versions{ARC 0.9}
% \begin{longtable}{llp{8cm}}
%    Options:&&\\
%    \texttt{-a, -all}& & all jobs\\
%    \texttt{-i, -joblist}& \textit{filename} & file containing a list of jobIDs\\
%    \texttt{-c, -cluster}&\verb#[-]#textem{name}&explicitly select or reject a specific site (cluster)\\
%    \texttt{-C, -clustlist}&\verb#[-]#textem{filename}&list of sites (clusters) to select or reject\\
%    \texttt{-s, -status}& \textit{statusstr} &only select jobs whose status is \textit{statusstr}\\
%    \texttt{-k, -kluster}&\verb#[-]#textem{name}&explicitly select or
%    reject a specific site (cluster) as re-submission target\\
%    \texttt{-K, -Klustlist}&\verb#[-]#textem{filename}&list of
%    clusters to select or reject as re-submission target\\
%    \texttt{-g, -giisurl}& \textit{url} &URL of a central Information System server\\
%    \texttt{-G, -giislist}&\textit{filename}&list of GIIS URLs\\
%    \texttt{-o, -joblist}&\textit{filename}&file where the job IDs will be stored\\
%    \texttt{~~~~-dryrun}&&add dryrun option to the xRSL\\
%    \texttt{~~~~-dumpxrsl}&&do not submit -- dump transformed xRSL to stdout\\
%    \texttt{~~~~-keep}& & keep files on gatekeeper (do not clean)\\
%    \texttt{-t, -timeout}& \textit{time} & timeout for queries (default 40 sec)\\
%    \texttt{-d, -debug}& \textit{debuglevel} &debug level, from -3 (quiet) to 3 (verbose) - default 0\\
%    \texttt{-x, -anonymous}& & use anonymous bind for queries (default)\\
%    \texttt{-X, -gsi}& & use GSI-GSSAPI bind for queries\\
%    \texttt{-v, -version}& & print version information\\
%    \texttt{-h, -help}& & print help page\\
%    Arguments:&&\\
%    \texttt{job ...} && list of job IDs and/or jobnames\\
% \end{longtable}


\subsection{arcclean}
\label{sec:arcclean}

If a job fails, or you are not willing to retrieve the results for
some reasons, a good practice for users is not to wait for the Grid
Manager to clean up the job leftovers, but to use \texttt{arcclean}
\index{arcclean}\index{commands:arcclean} to release the disk space
and to remove the job ID from the list of submitted jobs and from
the Information System.

\hspace*{0.5cm}
\begin{shaded}
   \uicommand{arcclean [options] [job ...]}
\end{shaded}
\versions{ARC 0.9}
\begin{longtable}{llp{8cm}}
   Options:&&\\
   \texttt{-a, --all}& & all jobs\\
   \texttt{-j, --joblist}& \textit{filename} & file containing a list of jobIDs\\
   \texttt{-c, --cluster}&\verb#[-]#textem{name}&explicitly select or reject a specific site (cluster)\\
   \texttt{-s, --status}& \textit{statusstr} &only select jobs whose status is \textit{statusstr}\\
   \texttt{-f, --force} & & removes the job ID from the local list even if the job is not found on the Grid\\
   \texttt{-t, --timeout}& \textit{time} & timeout for queries (default 20 sec)\\
   \texttt{-d, --debug}& \textit{debuglevel}&debug level is one of  FATAL, ERROR, WARNING, INFO, DEBUG or VERBOSE\\
   \texttt{-z, --conffile}&\textit{filename}& configuration file (default {\$}HOME/.arc/client.xml)\\
   \texttt{-v, --version}& & print version information\\
   \texttt{-h, --help}& & print help page\\
   Arguments:&&\\
   \texttt{job ...} && list of job IDs and/or jobnames\\
\end{longtable}

Only jobs that have finished can be cleaned.

% \subsection{arcrenew}
% \label{sec:arcrenew}
% 
% Quite often, the user proxy expires while the job is still running (or
% waiting in a queue). In case such job has to upload output files to a
% Grid location (Storage Element), it will fail. By using the \texttt{arcrenew}
% \index{arcrenew}\index{commands:arcrenew} command, users can upload
% a new proxy to the job. This can be done while a job is still running,
% thus preventing it from failing, or whithin 24 hours (or whatever is
% the expiration time set by the site) after the job
% end. In the latter case, the Grid Manager will attempt to finalize
% the job by uploading the output files to the desired location.
% 
% \hspace*{0.5cm}
% \begin{shaded}
%    \uicommand{arcrenew [options] [job ...]}
% \end{shaded}
% \versions{ARC 0.9}
% \begin{longtable}{llp{8cm}}
%    Options:&&\\
%    \texttt{-a, -all}& & all jobs\\
%    \texttt{-i, -joblist}& \textit{filename} & file containing a list of jobIDs\\
%    \texttt{-c, -cluster}&\verb#[-]#textem{name}&explicitly select or reject a specific site (cluster)\\
%    \texttt{-C, -clustlist}&\verb#[-]#textem{filename}&list of sites (clusters) to select or reject\\
%    \texttt{-s, -status}& \textit{statusstr} &only select jobs whose status is \textit{statusstr}\\
%    \texttt{-t, -timeout}& \textit{time} & timeout for queries (default 40 sec)\\
%    \texttt{-d, -debug}& \textit{debuglevel} &debug level, from -3 (quiet) to 3 (verbose) - default 0\\
%    \texttt{-x, -anonymous}& & use anonymous bind for queries (default)\\
%    \texttt{-X, -gsi}& & use GSI-GSSAPI bind for queries\\
%    \texttt{-v, -version}& & print version information\\
%    \texttt{-h, -help}& & print help page\\
%    Arguments:&&\\
%    \texttt{job ...} && list of job IDs and/or jobnames\\
% \end{longtable}
% 
% Prior to using \texttt{arcrenew}, be sure to actually create the
% new proxy!
% 
% \subsection{arcsync}
% \label{sec:arcsync}
% 
% If you are using User Interface installations on different machines,
% your local lists of submitted jobs will be different. To synchronise
% these lists with the information in the Information System, use the
% \texttt{arcsync} \index{arcsync}\index{commands:arcsync} command.
% 
% \hspace*{0.5cm}
% \begin{shaded}
%    \uicommand{arcsync [options]}
% \end{shaded}
% \versions{ARC 0.9}
% \begin{longtable}{llp{8cm}}
%    Options:&&\\
%    \texttt{-c, -cluster}&\verb#[-]#textem{name}&explicitly select or reject a specific site (cluster)\\
%    \texttt{-C, -clustlist}&\verb#[-]#textem{filename}&list of sites (clusters) to select or reject\\
%    \texttt{-g, -giisurl}& \textit{url} &URL of a central Information System server\\
%    \texttt{-G, -giislist}&\textit{filename}&list of GIIS URLs\\
%    \texttt{-f, -force}&&don't ask for confirmation\\
%    \texttt{-t, -timeout}& \textit{time} & timeout for queries (default 40 sec)\\
%    \texttt{-d, -debug}& \textit{debuglevel} &debug level, from -3 (quiet) to 3 (verbose) - default 0\\
%    \texttt{-x, -anonymous}& & use anonymous bind for queries (default)\\
%    \texttt{-X, -gsi}& & use GSI-GSSAPI bind for queries\\
%    \texttt{-v, -version}& & print version information\\
%    \texttt{-h, -help}& & print help page\\
% \end{longtable}
% 
% The ARC User Interface keeps a local list of jobs in the user's home
% directory (see section~\ref{sec:joblists}). If this file is lost,
% corrupt, or the user wants to recreate the file on a different
% workstation, the \texttt{arcsync} command will recreate this file from
% the information available in the Information System.

% \subsection{Auxilliary files}\label{sec:joblists}
% 
% User Interface keeps local job lists in two files:
% {\$}\verb#HOME/.ngjobs# and {\$}\verb#HOME/.arc/history#\footnote{In
%   ARC $\leq$ 0.5.48, {\$}\texttt{HOME/.nghistory}}.
% 
% \textbf{{\$}\texttt{HOME/.ngjobs}} is a local list of the user's active
% jobs. When a job is successfully submitted, it is added to this list,
% and when it is removed from the remote site, it is removed from this
% list. This list is used as the list of all active jobs when the user
% specifies \verb#-a# option to the various ARC user interface commands.
% For information about how to reconstruct this file in case it is
% damaged or you relocate to a different workstation, see
% section~\ref{sec:arcsync} about the \texttt{arcsync} command.
% 
% \textbf{{\$}\texttt{HOME/.arc/history}} contains the \texttt{jobID}s of
% the jobs the user has submitted together with the time of
% submission. This file is purely informational.
