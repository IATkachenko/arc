\chapter{Usage}

The Janitor can be used either with or without the A-REX service.
In case
A-REX is used, the invocation of the Janitor will be performed in an
automated manner.  It is then triggered by incoming jobs that request
a particular RTE for their execution.  If it is not already installed,
but
\begin{enumerate}
\item found as a MetaPackage in a Catalog that the site supports
\item with a package that first the BaseSystem of the site
\end{enumerate}
then it will be installed without further manual intervention by the
Janitor - triggered by the A-REX that received the compute request.

The Janitor's installation will not be affected by this decision
pro or cons and integration with \AREX. Both can be installed in
parallel. If \AREX is not allowed to install runtime environments upon
demand, such automated installations can still be invoked manually via
the Janitor's command line  interface.

\section{Janitor with A-REX}

Runtime Environments can be specified using the supported job description
languages.  The most representative two common languages shall be
explained at this point: xRSL and JSDL.  Listing~\ref{lst:xrsl_job}
shows the xRSL example in which two runtime environments are requested.

\lstsetXRSL
\begin{lstlisting}[
        label=lst:xrsl_job,
        caption={ [Job submission using the xRSL job description language.]
                  \textbf{Job submission using the xRSL job description language.}}
        ]
 &
 (executable = "run.sh" )
 (arguments = "weka.classifiers.trees.J48" "-t" "weather.arff")
 ("inputfiles" = ("weather.arff" "" ))
 ("stderr" = "stderr" )
 ("stdout" = "stdout" )
 ("gmlog" = "gmlog" )
 ("runtimeenvironment" = "APPS/BIO/WEKA-3.4.10")
 ("runtimeenvironment" = "APPS/BIO/WISE-2.4.1-5")
\end{lstlisting}
\task{The runtime environment names are composed out a directory name, the package name and the version number.}

A comprehensive reference manual of the Extended
Resource Specification Language (XRSL) can be found at
\href{www.nordugrid.org/documents/xrsl.pdf}{www.nordugrid.org/documents/xrsl.pdf}~\cite{NORDUGRID_MANUAL_4}.
Within Listing~\ref{lst:jsdl_job} an example using JSDL is provided. The
specification of how to assign runtime environments in JSDL is currently
only defined within the nordugrid jsdl-arc schema~
\href{http://svn.nordugrid.org/repos/nordugrid/arc1/trunk/src/services/a-rex/grid-manager/jobdesc/jsdl/jsdl_arc.xsd}
     {http://svn.nordugrid.org/repos/nordugrid/arc1/trunk/src/services/a-rex/grid-manager/jobdesc/jsdl/jsdl\_arc.xsd}.
\lstsetJUSTXML
\begin{lstlisting}[
        label=lst:jsdl_job,
        caption={ [Job submission using JSDL.]
                  \textbf{Job submission using JSDL.}}
        ]
<?xml version="1.0" encoding="UTF-8"?>
<JobDefinition
  xmlns="http://schemas.ggf.org/jsdl/2005/11/jsdl"
  xmlns:posix="http://schemas.ggf.org/jsdl/2005/11/jsdl-posix"
  xmlns:arc="http://www.nordugrid.org/ws/schemas/jsdl-arc">
  <JobDescription>
    <Application>
      <posix:POSIXApplication>
        <posix:Executable>/bin/sleep</posix:Executable>
        <posix:Argument>120</posix:Argument>
      </posix:POSIXApplication>
    </Application>
    <DataStaging>
      <FileName>test.sh</FileName>
      <Source/>
      <Target/>
    </DataStaging>
    <DataStaging>
      <FileName>transferGSI-small</FileName>
      <Source>
        <URI>gsiftp://pgs02.grid.upjs.sk:2811/unixacl/transferGSI-small</URI>
      </Source>
      <Target/>
    </DataStaging>
    <Resources>
      <arc:RunTimeEnvironment>
        <arc:Name>APPS/BIO/WISE-2.4.1-5</arc:Name>
        <arc:Version><Exact>2.4.1</Exact></arc:Version>
      </arc:RunTimeEnvironment>
      <arc:RunTimeEnvironment>
        <arc:Name>APPS/BIO/APPS/BIO/WEKA-3.4.10</arc:Name>
        <arc:Version><Exact>3.4</Exact></arc:Version>
      </arc:RunTimeEnvironment>
    </Resources>
  </JobDescription>
</JobDefinition>
\end{lstlisting}


\section{Janitor without A-REX}

On Linux systems, the Janitor's standalone commandline tool is
available as /usr/lib/arc/janitor.  Some Linux distributions may
prefer /usr/libexec or similar paths.  The script is only functional as
root\footnote{Should you find that constraint unbearable for your purpose,
please investigate the file rjanitor.cc in the ARC source tree. It wraps
the janitor application and as a C binary can be configured to attract
root privileges.}.  To find that binary directly, you may decide to add
that location to your \$PATH environment variable.

The available commands to the Janitor, implemented as options to the janitor script, are listed
in the Table~\ref{tab:janitor_commandline_man}.

\begin{table}[!h]
   \begin{center}
        \mycaption{Overview about the available commands to the Janitor.}{}
        \label{tab:janitor_commandline_man}
	\begin{tabular}{p{0.5cm}p{2cm}p{11cm}}
	\multicolumn{3}{l}{\textbf{janitor [COMMAND] [JOB-ID] [RTE] \dots}} \\
	\multicolumn{3}{l}{\textbf{Command:}}\\
	&	register & Registers a job and a set of runtime
			   environments in the Janitor database.
			   Requires the parameters [JOB-ID] and a list of [RTE]s.\\
	&	deploy	 & Downloads and installs the desired
			   runtime environments. Requires the name
			   of an already registered [JOB-ID].\\
	&	remove	 & Removes the placeholder of the job on the runtime environments.
			   If no more jobs are using the runtime environment and the
			   lifespan of the runtime environment has be expired, the runtime
			   environment can be removed using the \texttt{sweep} command.
			   Requires the [JOB-ID] to be removed.\\
	&		 &\\
	&	sweep	 & Removes unused runtime environments. No further arguements are required.
			   Using the option \texttt{--force} enforces the removal of all unused
			   runtime environments. Runtime environments having the state FAILED will
			   not be removed.\\
	&	setstate & Changes the state of a dynamically installed runtime environment.
			   This might be useful in case a runtime environment with a state FAILED
			   shall be removed (new state might be REMOVAL\_PENDING). Requires the argument
			   [STATE] followd by a list of [RTE]s.\\
	&		 &\\
	&	search	 & Performs a simple search in the catalog and the manually installed
			   runtime environments (\texttt{runtimedir}). Requires no [JOB-ID] nor [RTE]s,
			   but only a list of string to be searched for.\\
	&	list	 & Lists all information about jobs, automatically installed runtime
			   environments and manually installed runtime environments.
			   No additional parameters have to be passed.\\
	&	info	 & Renders information about a job. Requires the parameter [JOB-ID].\\
	\multicolumn{3}{l}{\textbf{Job id:}}\\
	&		 & A unique sequence of numbers. Once Janitor
			   registered a job id, it cannot register a
			   second job having the same job id.\\
	\multicolumn{3}{l}{\textbf{Runtime environments:}}\\
	&		& Runtime environments are defined by a continuous
			  string.  The name of valid runtime environment
			  names can be investigated using
			  the \texttt{list} or the \texttt{search}
			  commands. They are defined in the
			  catalog or by the directories and
			  scripts of the \texttt{runtimedir}
			  of the \texttt{grid-manager}.\\
	\end{tabular} 
   \end{center}
\end{table}

The most important commands for the Janitor are \texttt{register},
\texttt{deploy} and \texttt{remove}. To register a job along with a set
of runtime environments in the Janitor, the first command \texttt{register}
followed by a job identifier and a list of runtime environments has to be
used.  A job is identified by a sequence of numbers. Runtime environments
are specified by a string containing the name as it is defined within
the Catalog (resp. the runtime directory of the grid-manager).  The command
\texttt{deploy} extracts the necessary dependencies of the desired
dRTEs and then downloads and installs the required packages.

%How dependencies can are defined will be explained later in the section~\ref{sec:catalog}.

In order to remove jobs registered in the Janitor, the command
\texttt{remove} has to be used.  The command only removes the job
entry and the lock on the runtime environment. If there are no more
locks on the runtime environment it is ok to be deleted also physically
from the disk.  The demand to pass a job number for the removal of a
RTE is irritating at first.  This shall prevent the removal of runtime
envrironments that are still being used by jobs in the system. Instead,
the janitor is informed about a job's termination and is requested to
remove the assignment of that job to the runtime environment. Only those
RTEs with no job-assignment are eligible for being sweeped. RTEs come
with an expiry time or the command may be performed via the command line.

Easy command line examples are provided in Listing~\ref{lst:janitor_example}.
You may also want to inspect the janitor(8) man page.

Every command has a certain behaviour for its exit status.
Table~\ref{tab:janitor_commandline_exit_status} lists the possible
outcomes. A value of 0 always indicates that no error occurred.

\begin{table}[!h]
   \begin{center}
        \mycaption{Possible exit states of the janitor application}{}
        \label{tab:janitor_commandline_exit_status}
	\begin{tabular}{p{0.5cm}p{2cm}p{0.5cm}p{11cm}}
	\multicolumn{3}{l}{\textbf{Exit status:}} \\
	&\multicolumn{3}{l}{The exit status of Janitor depends on the used command.} \\
	&	register			& 0 & Registration was successful. No noteworthy occurrences.\\
	&					& 1 & Registration was successful but some runtime environments aren't installed yet. Deploy is mandatory.\\ 
	&					& 2 & An error occured.\\
	&					&   &\\
	&	deploy				& 0 & Sucessfully initialized job.\\
	&					& 1 & Can't provide requested runtime environments.\\ 
	&					&   &\\
	&	remove				& 0 & Sucessfully removed job or no such job.\\
	&					& 1 & Can't provide requested runtime environments.\\ 
	&					&   &\\
	&	sweep				& 0 & Always returns this exit code.\\
	&					&   &\\
	&	setstate			& 0 & Changing the state was successful.\\
	&					& 1 & Can not change the state.\\ 
	&					&   &\\
	&	search				& 0 & Search sucessfully finished.\\
	&					&   &\\
	&	list				& 0 & Successfully retrieved information.\\
	&					&   &\\
	&	info				& 0 & Successfully retrieved job information.\\
	&					& 1 & No such job.\\ 
	&					& 2 & Error while retrieving job information.\\
	\end{tabular} 
   \end{center}
\end{table}


\lstsetKSH
\begin{lstlisting}[
        label=lst:janitor_example,
        caption={ [Example \textit{log.conf} settings for janitor.]
                  \textbf{Example \textit{log.conf} settings for janitor.}}
        ]
# janitor register 1999 APP/BIO/JASPAR-CORE-1.0 APPS/BIO/APPS/BIO/WEKA-3.4.10
# janitor deploy 1999
# janitor remove 1999

# janitor sweep --force
# janitor setstate REMOVAL_PENDING APP/BIO/JASPAR-CORE-1.0 APPS/BIO/APPS/BIO/WEKA-3.4.10

# janitor search JASPAR WEKA
# janitor list
# janitor info 1999
\end{lstlisting}

Once a dynamic runtime environment is installed, it looks completely
indistinguishable from traditionally installed runtime environments. This
also means that the general concept to have one installation performed
for all compute nodes in the network is kept.

% 
% * Usage
% ** Without A-REX
% *** Commandline (Advantage/Disadvantages)
% *** WebService (Advantage/Disadvantages)
% 
% * With A-REX
% ** Example:
% *** Simple:     runtimedir
% *** Simple:     RDF
% *** Simple job: JSDL
% *** Simple job: XRSL



%\section{Example}
% 
% 

\section{Janitor with \AREX}

The motivation to have a runtime environment available comes from the submitters
of the grid jobs that depends on that runtime environment for their execution.
The site administrator's sole responsibility is to have the dynamic runtime
environment at the site's disposal. No more. With \AREX allowed to initiated
the commands to the Janitor, no further interaction from the site administrator
is required. An exception may be to confirm the consistency of the system when
the machine has crahsed and the Janitor may still find jobs assinged to runtime
environments that are no longer running.

Another exception for an active involvement of the site administrator is the
initial configuration of the Janitor and the updating of runtime environments
that are eligible to be installed.
