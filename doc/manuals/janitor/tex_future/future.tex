\chapter{Outlook}
% 
% * Future work
% ** API and WebService-Interface for Janitor
% ** Glue2 specification for RTE states
% 

\section{Representation of dynamic REs in the information model}
Dynamic REs require an extended representation in the
information model.  The Application Software description should be able
to distinguish installed REs from installable REs, potentially offer
description of extended RE state-like information. This work is planned
to be carried out as part of the Glue-2.0 effort of OGF\footnote{OGF
GLUE: https://forge.gridforum.org/sf/projects/glue-wg}.

\section{Integration with Workflow Management}

Future development of ARC aims at integrating grid computing with
workflow tools for the web services that have a growing user base in
bioinformatics. The challenge is to prepare REs for
programs or databases and to offer such concisely to users of the
workflow environments.  In the bioinformatics community, such are
today offered as web services.  This anticipated development instead
fosters the dynamic installation on the grid whenever appropriate to
allow for special computational demands in high-throughput analyses.
Conversely, because of the increased complexity of workflows with respect
to the already today not manually manageable number of REs, without an
automatism for the automated installation of software packages on the
grid, the use of workflows in grid computing seems mute.

\section{Implementation of a Catalog service}
 
A Catalog service is planed to be implemented on top of the ARC
HED component.  This service will render the currently used locally
accessible RDF file externally accessible. Selected users are then allowed
to remotely add/edit/remove REs to/from to it.  The Janitor will access
the content of the Catalog through a well-defined Web Service interface.

\section{Integration with the Virtualization work}

The RDF schema nicely prepares for the upcoming virtualisation of worker
nodes. Hereto, the {\tt BaseSystem} indicates a virtual image to which
further packages, the dynamic REs, would then be added.
How exactly the dynamics are integrated will depend on how
dynamic the virtualisation of the nodes is. In the simplest scenario,
a worker node's CPU will only be occupied by a single virtual machine
and that will not be changed. In this case, there is no difference to
the setup of the Janitor with today's static setups.

However, if the BaseSystems can be substituted dynamically, then a RE
can possibly be offered via multiple BaseSystems. The RDF
Schema describes BaseSystems as separate instances and as such differs
from the current RE registry.  Heuristics that prefer one BaseSystem for
another can make direct use of the data that is presented in the schema.
The integration of packages from Linux distributions in the description
of REs is essential to have a means to decide for the equivalence of
manual additions and the functionality that comes with BaseSystem.


TODO:\\
\begin{itemize}
 \item There is a arc.conf file in which all possiblie flags are listed... ADD THE JANITOR FLAGS!!!
 \item Dynamic RTEs are now listed as manually installed in the ``janitor list'' command.. change that
 \item Verfication
\end{itemize}
