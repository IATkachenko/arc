\usepackage[pdftex]{graphicx}

%\usepackage{ifthen}



% Macht beim fett schreiben auch das Mathe-Zeug fett
%
\newcommand{\allbf}[1]{\textbf{\boldmath#1\unboldmath}}

% FANCY CAPTION
% \caption[#1]{\textbf{#1}#1}
\newcommand{\mycaption}[2]{\caption[#1]{\textbf{\boldmath#1\unboldmath} #2}}







%%%
%%% Now you can use \ifcolor as follows
%%% \ifcolor
%%%    Text parsed by PDFLaTeX
%%% \else
%%%    Text parsed if PDFLaTeX is not used
%%% \fi
%%% 
\newif\ifcolor\colortrue



%
%   COLOR
%
%
\usepackage{color}
\definecolor{brightred}{rgb}{1,0.9,0.9}
\definecolor{brightgreen}{rgb}{0.9,1,0.9}
\definecolor{brightblue}{rgb}{0.95,0.95,1}
\definecolor{brightyellow}{rgb}{1,1,0.85}

\definecolor{lightred}{rgb}{1,0.75,0.75}
\definecolor{lightgreen}{rgb}{0.75,1,0.75}
\definecolor{lightblue}{rgb}{0.75,0.75,1}
\definecolor{lightyellow}{rgb}{1,1,0.75}

\definecolor{red}{rgb}{1,0,0}
\definecolor{green}{rgb}{0,1,0}
\definecolor{blue}{rgb}{0,0,1}

\definecolor{darkred}{rgb}{.7,0,0}
\definecolor{darkgreen}{rgb}{0,.7,0}
\definecolor{darkblue}{rgb}{0,0,.7}

\definecolor{white}{rgb}{1,1,1}
\definecolor{lightgray}{rgb}{0.95,0.95,0.95}
\definecolor{gray}{rgb}{0.5,0.5,0.5}
\definecolor{black}{rgb}{0,0,0}

\definecolor{marker}{rgb}{0.9,0.9,0.9}
\definecolor{urgent}{rgb}{1,0,0}
\definecolor{discreeturgent}{rgb}{1,0.5,0.5}
\definecolor{discreetcomment}{rgb}{0.25,0.9,0.25}
\definecolor{checked}{rgb}{0,.7,0}



%\newif\ifshowtasks\showtaskstrue
\newif\ifshowtasks\showtaskstrue
\newcommand{\task}[1]{\ifshowtasks\textcolor{discreeturgent}{~#1~}\else\fi}
\newcommand{\comment}[1]{\ifshowtasks\textcolor{discreetcomment}{~#1~}\else\fi}
\newcommand{\drain}[1]{}

%
%    LISTINGS
% \begin{lstlisting}[float,caption=A floating example]
%    for i:=maxint to 0 do
%    begin
%      { do nothing }
%    end;
%    Write(’Case insensitive ’);
%    WritE(’Pascal keywords.’);
% \end{lstlisting}
%
% for referencing line numbers use (*@\label{comment}@*)  inside the lstlistings enviroment
\usepackage{listings}
\ifcolor
          \lstset{							% general command to set parameter(s)
               escapeinside={(*@}{@*)},
               xrightmargin=0.5cm,					% for centering
               xleftmargin=1.5cm,					% .. \textwidth shrinks after the first time, which is stupid
               framexleftmargin=20pt,					% for having the numbers beneath the h rules
               framexrightmargin=0pt,
               framextopmargin=2ex,					% draw good looking space between the lines
               framexbottommargin=2ex,					%... and the listing
               frame=single,						% h rules top and bottom
               language=C,
               tabsize=4,
               numbers=left,
               numberstyle=\footnotesize, 
               numbersep=8pt,
               basicstyle=\ttfamily\scriptsize, 		% print whole listing small
               breaklines=true,
               keywordstyle=\color{black}\bfseries,		% underlined bold black keywords
               identifierstyle=,				% nothing happens
               commentstyle=\color{darkblue}\itshape,		% white comments
               stringstyle=\ttfamily, 				% typewriter type for strings
               morekeywords=[2]{and,or,not},
               emph={wichtiges,zeug},				% additional keywords
               emphstyle=\underbar,
               showstringspaces=false} 				% no special string spaces
\else
          \lstset{						% general command to set parameter(s)
               escapeinside={(*@}{@*)},
               xrightmargin=0.5cm,				% for centering
               xleftmargin=1.5cm,
               framexleftmargin=20pt,				% for having the numbers beneath the h rules
               framexrightmargin=0pt,
               framextopmargin=2ex,				% draw good looking space between the lines
               framexbottommargin=2ex,				%... and the listing
               frame=single,					% h rules top and bottom
               language=C,
               tabsize=4,
               numbers=left,
               numberstyle=\footnotesize, 
               numbersep=8pt,
               basicstyle=\scriptsize\ttfamily,			% print whole listing small
               breaklines=true,
               keywordstyle=\color{black}\bfseries,		% underlined bold black keywords
               identifierstyle=,									% nothing happens
               commentstyle=\color{black}\itshape,			% white comments
               stringstyle=\ttfamily, 							% typewriter type for strings
               morekeywords=[2]{and,or,not},
               emph={wichtiges,zeug},								% additional keywords
               emphstyle=\underbar,
               showstringspaces=false} 						% no special string spaces
\fi

%
\usepackage[labelfont={bf},font=small]{caption,subfig} 
% justification=raggedright,format=hang,labelfont={bf},font=small
% justification=justified,singlelinecheck=false

%
%    HYPERREF
%
%    (depends on z_layout_settings color)
%
% NO HANDLING YET FOR NOT PDF!!!
\ifcolor
  \usepackage[pdftex,
            colorlinks=true, linkcolor=blue, urlcolor=blue, citecolor=blue,
            raiselinks=true,
            bookmarks=false,
            bookmarksopenlevel=1,
            bookmarksopen=true,
            bookmarksnumbered=true,
            hyperindex=true,
            plainpages=false,% correct hyperlinks
            pdfpagelabels=true%,% view TeX pagenumber in PDF reader
            %pdfborder={0 0 0.5}
            ]{hyperref} % erzeuge Hyperlinks z.B. für pdflatex
\else
  \usepackage[pdftex,
            colorlinks=true, linkcolor=black, urlcolor=black, citecolor=black,
            raiselinks=true,
            bookmarks=false,
            bookmarksopenlevel=1,
            bookmarksopen=true,
            bookmarksnumbered=true,
            hyperindex=true,
            plainpages=false,% correct hyperlinks
            pdfpagelabels=true%,% view TeX pagenumber in PDF reader
            %pdfborder={0 0 0.5}
            ]{hyperref} % erzeuge Hyperlinks z.B. für pdflatex
\fi





\newcommand{\lstsetCPP}{%
\ifcolor%
\lstset{
	escapeinside={//(@*}{*@)},
	language=C++,
	frame=single,
	numbers=left,
	backgroundcolor=\color{brightblue}
	}%
\else%
\lstset{
	escapeinside={//(@*}{*@)},
	language=C++,
	frame=single,
	numbers=none,
	backgroundcolor=\color{white}
	}%
\fi%
}

\newcommand{\lstsetARCHEDXML}{%
\lstset{
	escapeinside={<!--(@*}{*@)-->},
	language=XML,
	frame=single,
	numbers=left,
	backgroundcolor=\color{lightyellow}
	}%
}

\newcommand{\lstsetJUSTXML}{%
\lstset{
	escapeinside={<!--(@*}{*@)-->},  % <!--(*@\label{comment}@*)-->
	language=XML,
	frame=single,
	numbers=left,
	backgroundcolor=\color{brightyellow}
	}%
}


\newcommand{\lstsetKSH}{%
\lstset{
	escapeinside={(@*}{*@)},
	language=ksh,
	frame=single,
	numbers=none,
	backgroundcolor=\color{lightgray}
	}%
}

% Sometimes latex just won't accept a line break.
% With this command you can do it anyway! HAR HAR HAR
\newcommand{\forcelinebreak}{
%\vspace{\bigskipamount}
\hspace*{\fill} \\
} 


\usepackage{framed}                                %for shaded and framed paragraphs
\usepackage{textcomp}                              %for various symbols, e.g. Registered Mark
%
\def\efill{\hfill\nopagebreak}%
\hyphenation{Nordu-Grid}
\setlength{\parindent}{0cm}
\setlength{\FrameRule}{1pt}
\setlength{\FrameSep}{8pt}
\addtolength{\parskip}{5pt}
\renewcommand{\thefootnote}{\fnsymbol{footnote}}
\renewcommand{\arraystretch}{1.3}
\newcommand{\dothis}{\colorbox{shadecolor}}
\newcommand{\globus}{Globus Toolkit\textsuperscript{\textregistered}~2~}
\newcommand{\GT}{Globus Toolkit\textsuperscript{\textregistered}}
\newcommand{\ngdl}{\url{http://ftp.nordugrid.org/download}~}
\definecolor{shadecolor}{rgb}{1,1,0.6}
\definecolor{salmon}{rgb}{1,0.9,1}
\definecolor{bordeaux}{rgb}{0.75,0.,0.}
\definecolor{cyan}{rgb}{0,1,1}
%
%----- DON'T CHANGE HEADER MATTER