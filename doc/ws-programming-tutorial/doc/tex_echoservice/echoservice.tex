\chapter{The echo service}

The next service which will be presented is the echo service. It simply returns the received message back to the client.
Depending on the operation requested by the client, the letters of the message will be reordered reverse or not.
The goal of this chapter is to offer a deeper knowledge in message processing. Furthermore the \textit{WSDL} (Web Services Description Language) shall be introduced along with this example.

\section{Web Services Description Language}

Web Services are often reachable for a large set of users in order to easily access complex information. In many cases the Web Service are capable to replace HTML pages i.e. access to databases like Pubmed or the NLM catalog.
% http://www.ncbi.nlm.nih.gov/entrez/query/static/esoap_help.html
In contrary to HTML the interfaces of the data are well defined and easier to find for machines. bla...
Due to that reason, the way to access the web service has to be defined in an own language: \textit{WSDL} (Web Services Description Language).
WSDL is a meta language which determines the structure of the messages along with its permitted elements, the accepted operations, the supported protocols and the address to reach the service. 
The WSDL for the echo service which shall be implemented in this chapter is to be seen in Listing~\ref{lst:echo_wsdl}. There are five main elements to describe a service:
\begin{itemize}
	\item \textbf{Types} --- Definition of the data types used by the messages, line~\ref{lst_code:echo_wsdl_types}.
	\item \textbf{Message} --- Assignments which data types are representing a message, line~\ref{lst_code:echo_wsdl_message1} and \ref{lst_code:echo_wsdl_message2}.
	\item \textbf{Port type} --- \parbox[t]{13cm}{Assignment of the interface: One-way (input), request-response (input, output), solicit-response (input, output, fault), notification (output). The request-response interface is used in line~\ref{lst_code:echo_wsdl_portType}.}
	\item \textbf{Binding} --- Defines the concret protocol and data format used for the message transmission, line~\ref{lst_code:echo_wsdl_binding}.
	\item \textbf{Service} --- Specifies the address to bind a service, line~\ref{lst_code:echo_wsdl_service}.
\end{itemize}
%\textcolor{white}{newline}
\lstsetJUSTXML
\lstinputlisting
	[
	label=lst:echo_wsdl,
	caption={[WSDL file describing the echo service. Filename: echo.wsdl]
	\textbf{HWSDL file describing the echo service. Filename: echo.wsdl}}
	]
{../src/services/echoservice/echo.wsdl}


More information about WSDL may be found at \href{http://www.w3.org/TR/wsdl}{http://www.w3.org/TR/wsdl}.

% a valid service request and the  


% WSDL ist eine Metasprache, mit deren Hilfe die angebotenen Funktionen, Daten, Datentypen und Austauschprotokolle eines Web Service beschrieben werden können. Es werden im Wesentlichen die Operationen definiert, die von außen zugänglich sind, sowie die Parameter und Rückgabewerte dieser Operationen. Im Einzelnen beinhaltet ein WSDL-Dokument funktionelle Angaben zu:

%    * der Schnittstelle
%    * Zugangsprotokoll und Details zum Deployment
%    * Alle notwendigen Informationen zum Zugriff auf den Service, in maschinenlesbarem Format

%Nicht enthalten sind hingegen:

%    * Quality-of-Service-Informationen
%    * Taxonomien/Ontologien zur semantischen Einordnung des Services

\section{Service}

The service defined by the WSDL file in the previous section shall now be implemented. The source code of the C++ file is shown in Listing~\ref{lst:echo_service_cpp}. Again the header file will be set aside for it contains to much redundant information.

\lstinputlisting
	[
	label=lst:echo_service_cpp,
	caption={[C++ implementation of the echo service. Filename: echoservice.cpp]
	\textbf{C++ implementation of the echo service. Filename: echoservice.cpp}}
	]
{../src/services/echoservice/echoservice.cpp}


\lstsetARCHEDXML

\lstinputlisting
	[
	label=lst:arcecho_arched_xml, float=htb,
	caption={[HED configuration file for the Arc intern echo service. Filename: arcecho\_no\_ssl.xml]
	\textbf{HED configuration file for the Arc intern echo service. Filename: arcecho\_no\_ssl.xml\textcolor{white}{hmf}}}
	]
{../src/services/echoservice/arched_echoservice.xml}





\lstsetKSH
\begin{lstlisting}[
label=lst:invokation_arched_timeservice,float=htb,
caption={[Transformation in eine uniforme konzentrische Verteilung.]
         \textbf{Transformation in eine uniforme konzentrische Verteilung.\textcolor{white}{hmf}}}]
$ arched -c arched_echoservice.xml  && echo jo ||echo n
\end{lstlisting}



\section{Client}


\lstsetCPP
\lstinputlisting
	[
	label=lst:arcecho_arched_xml,
	caption={[HED configuration file for the Arc intern echo service. Filename: arcecho\_no\_ssl.xml]
	\textbf{HED configuration file for the Arc intern echo service. Filename: arcecho\_no\_ssl.xml\textcolor{white}{hmf}}}
	]
{../src/clients/echoclient/echoclient.cpp}

\lstsetCPP




\lstsetKSH
\begin{lstlisting}[
label=lst:invokation_arched_timeservice, float=htb,
caption={[Transformation in eine uniforme konzentrische Verteilung.]
         \textbf{Transformation in eine uniforme konzentrische Verteilung.\textcolor{white}{hmf}}}]
$ ./echoclient http://localhost:60000/echo ordinary text_to_be_transmitted
[ text_to_be_transmitted ]
$ ./echoclient http://localhost:60000/echo reverse text_to_be_transmitted
[ dettimsnart_eb_ot_txet ]
\end{lstlisting}





\lstsetJUSTXML
\begin{lstlisting}[
label=lst:timeservice_cpp_source, float=htb,
caption={[Transformation in eine uniforme konzentrische Verteilung.]
         \textbf{Transformation in eine uniforme konzentrische Verteilung.\textcolor{white}{hmf}}}]
<soap-env:Envelope xmlns:echo="urn:echo" xmlns:soap-enc="http://schemas.xmlsoap.org/soap/encoding/" xmlns:soap-env="http://schemas.xmlsoap.org/soap/envelope/" xmlns:xsd="http://www.w3.org/2001/XMLSchema" xmlns:xsi="http://www.w3.org/2001/XMLSchema-instance">
  <soap-env:Body>
    <echo:echoRequest>
      <echo:say operation="reverse">text_to_be_transmitted</echo:say>
    </echo:echoRequest>
  </soap-env:Body>
</soap-env:Envelope>
\end{lstlisting}



\lstsetJUSTXML
\begin{lstlisting}[
label=lst:timeservice_cpp_source, float=htb,
caption={[Transformation in eine uniforme konzentrische Verteilung.]
         \textbf{Transformation in eine uniforme konzentrische Verteilung.\textcolor{white}{hmf}}}]
<soap-env:Envelope xmlns:echo="urn:echo" xmlns:soap-enc="http://schemas.xmlsoap.org/soap/encoding/" xmlns:soap-env="http://schemas.xmlsoap.org/soap/envelope/" xmlns:xsd="http://www.w3.org/2001/XMLSchema" xmlns:xsi="http://www.w3.org/2001/XMLSchema-instance">
  <soap-env:Body>
    <echo:echoResponse>
      <echo:hear>[ dettimsnart_eb_ot_txet ]</echo:hear>
    </echo:echoResponse>
  </soap-env:Body>
</soap-env:Envelope>
\end{lstlisting}









Folgender XML Aufruf und Antwort soll \textit{automatisch} generiert werden:

%[caption={[]WegDamit},language=XML,basicstyle=\scriptsize,breaklines=true,label=lst:request] 





Zertifikate
Zustände (Nutzer wiedererkennen, arbeit aufnehmen)
