\chapter{The Echo Web Service}

The next service to be presented is the Echo Web Service.
Its task is to return the content of the received message back to the client.
Furthermore, the service provides two additional operations called ``ordinary'' and ``reverse''.
Depending on the operation chosen by the client, the message gets returned unmodified or reversed.
The goal of this chapter is to offer a deeper knowledge about message processing.
Also, the \textit{WSDL} (Web Services Description Language) will be introduced, with which the section now starts.

\section{Web Services Description Language}

Web Services can be specified (not implemented) by a meta language called \textit{WSDL} (Web Services Description Language).
Beside the description of the service, WSDL renders the possibility that services can be discovered by service brokers.
The discovery of services may be useful for an automatic adaption of a running client, i.e. if the endpoint of the service changes or to share the load of demands.
The WSDL service specification enables the programmer to implement a suitable client which interfaces the service in a proper way.
WSDL describes the structure of the messages, the intended use of the messages, the supported protocols and the endpoint of the service (URL).
\task{For that last reason, every instance of a service must be produced a distinct WSDL file for - or - it is automatically prepared from the information that the server has available once it is implemented. - ?MG }\\

The WSDL file that specifies the here presented Echo Web Service is shown in listing~\ref{lst:echo_wsdl}. 
It is composed of five main elements:
\newcommand{\parboxWidth}{13cm}
\begin{itemize}
	\item \textbf{Types} --- The element \textit{types} contains the description of the message structures and is written in the language XSD (XML Schema Definition). The data types used for the messages of the Echo Web Service are defined starting with the  line~\ref{lst_code:echo_wsdl_types}.

	\item \textbf{Message} --- Within the element \textit{message} a subset of the previously defined types are assigned to be a message. In case of the Echo Web Service  two messages are designated: echoRequest and echoResponse, compare lines line~\ref{lst_code:echo_wsdl_message1} and \ref{lst_code:echo_wsdl_message2}.

	\item \textbf{Port type} --- The element \textit{portType} is used to assign the messages to an interface. Four interface types can be distinguished: One-way (input), request-response (input, output), solicit-response (input, output, fault) and notification (output).
	The echo service realises a solicit-response interface which is declared subsequent to line~\ref{lst_code:echo_wsdl_portType}. % in the lines followed by the line X

	\item \textbf{Binding} --- The protocol and the data format used for the message transmission are denoted in the element \textit{binding}. Possible style attributes are \textit{document} or \textit{rpc} (Remote Procedure Call). The transport attribute defines the protocol which is regularly HTTP. For each operation a corresponding SOAP action has to be defined which assignes a messages to be either \textit{literal} or \textit{encoded}~\cite{BUTEK_2009}. The service implemented in this chapter will use the binding \textit{document}/\textit{literal} and the protocol HTTP which is to be seen in the lines followed by  line~\ref{lst_code:echo_wsdl_binding}.

% The style attribute can be "rpc" or "document". In this case we use document. The transport attribute defines the SOAP protocol to use. In this case we use HTTP.
%The operation element defines each operation that the port exposes.
%For each operation the corresponding SOAP action has to be defined. You must also specify how the input and output are encoded. In this case we use "literal".
%  http://www.ibm.com/developerworks/webservices/library/ws-whichwsdl/
%   1. RPC/encoded    (Remote Procedure Call)
%   2. RPC/literal
%   3. Document/encoded
%   4. Document/literal
% The terminology here is very unfortunate: RPC versus document. These terms imply that the RPC style should be used for RPC programming models and that the document style should be used for document or messaging programming models. That is not the case at all. The style has nothing to do with a programming model. It merely dictates how to translate a WSDL binding to a SOAP message. Nothing more. You can use either style with any programming model.
%
% RPC/literal SOAP message for myMethod
%
%<soap:envelope>
%    <soap:body>
%        <myMethod>
%            <x>5</x>
%            <y>5.0</y>
%        </myMethod>
%    </soap:body>
%</soap:envelope>
%
%  RPC/encoded SOAP message for myMethod
% 
% <soap:envelope>
%     <soap:body>
%         <myMethod>
%             <x xsi:type="xsd:int">5</x>
%             <y xsi:type="xsd:float">5.0</y>
%         </myMethod>
%     </soap:body>
% </soap:envelope>

	\item \textbf{Service} --- Within the last WSDL element \textit{service} the endpoint of the service gets specified. As to be seen in line~\ref{lst_code:echo_wsdl_service} the endpoint of the Echo Web Service  is assigned to \textit{http://localhost:60000/echo}.

\end{itemize}
%\textcolor{white}{newline}
\lstsetJUSTXML
\lstinputlisting
	[
	label=lst:echo_wsdl,float=p,
	caption={[WSDL file describing the echo service. Filename: echo.wsdl]
	\textbf{WSDL file describing the echo service. Filename: echo.wsdl}}
	]
{../src/services/echoservice/echo.wsdl}
\task{Validate WSDL}


\subsection{The SOAP Fault Element}

The optional SOAP Fault element is used to indicate error messages in case of the port type \textit{solicit-response} (input, output, fault).
If a \textit{Fault} element is present, it must appear as a child element of the \textit{Body} element. A \textit{Fault} element can only appear once in a SOAP message.

The SOAP Fault element has the following sub elements:
\begin{tabular}{lp{9cm}}
Sub Element 	& Description \\
<faultcode> 	& A code for identifying the fault \\
<faultstring> 	& A human readable explanation of the fault \\
<faultactor> 	& Information about who caused the fault to happen \\
<detail> 	& Holds application specific error information related to the Body element\\
\end{tabular}


SOAP Fault Codes

The faultcode values defined below must be used in the faultcode element when describing faults:
\begin{tabular}{lp{9cm}}
Error           & Description \\
VersionMismatch & Found an invalid namespace for the SOAP Envelope element \\
MustUnderstand  & n immediate child element of the Header element, with the mustUnderstand attribute set to "1", was not understood \\
Client          & he message was incorrectly formed or contained incorrect information \\
Server          & here was a problem with the server so the message could not proceed \\
\end{tabular}

\task{Source of the above two tables? Extension of the second table? Numbers??}

More information about WSDL may be found at \href{http://www.w3.org/TR/wsdl}{http://www.w3.org/TR/wsdl}.

% a valid service request and the  


% WSDL ist eine Metasprache, mit deren Hilfe die angebotenen Funktionen, Daten, Datentypen und Austauschprotokolle eines Web Service beschrieben werden können. Es werden im Wesentlichen die Operationen definiert, die von außen zugänglich sind, sowie die Parameter und Rückgabewerte dieser Operationen. Im Einzelnen beinhaltet ein WSDL-Dokument funktionelle Angaben zu:

%    * der Schnittstelle
%    * Zugangsprotokoll und Details zum Deployment
%    * Alle notwendigen Informationen zum Zugriff auf den Service, in maschinenlesbarem Format

%Nicht enthalten sind hingegen:

%    * Quality-of-Service-Informationen
%    * Taxonomien/Ontologien zur semantischen Einordnung des Services
\clearpage

\section{Service}

This section presents the implementation of the service defined in the WSDL file of the previous section.
The source code of the C++ file is shown in listing~\ref{lst:echo_service_cpp}.
The header file will be set aside for it contains only redundant information. 
It can be reread in the source directory that is accompanying the tutorial.
Regarding to the source code of the Echo Web Service basically three changes of the Time Web Service source code will be required.
The first change concerns the constructor and can be found at line~\ref{lst_code:echo_cpp_prefix}.
Two strings are extracted of the server configuration file which later are used to envelope the message before it is returned.
Furthermore, a new method \textit{makeFault} in line~\ref{lst_code:echo_cpp_makeFault} has been introduced which is able to create the fault messages and is called by the method \textit{process}.
Within the method \textit{process} two new code fragments are now extracting and analyzing the incoming message, to be seen at line~\ref{lst_code:echo_cpp_extracting} and \ref{lst_code:echo_cpp_analyzing}.
In case something unexpected happens both will create a SOAP fault message.\\
% Vielleicht noch etwas über das extrahieren der Nachricht schreiben

\lstsetCPP
\lstinputlisting
	[
	label=lst:echo_service_cpp,
	caption={[C++ implementation of the echo service. Filename: echoservice.cpp]
	\textbf{C++ implementation of the echo service. Filename: echoservice.cpp}}
	]
{../src/services/echoservice/echoservice.cpp}


Again, the HED is configured by an XML file and is launched as introduced earlier.
The server configuration file is shown in Listing~\ref{lst:echo_hed_configuration_xml}.
The only change concerns the lines~\ref{lst_code:echo_hed_configuration_xml_pre} and \ref{lst_code:echo_hed_configuration_xml_pre} which specify the \textit{prefix} and \textit{suffix}.
The parameters will be extracted in the constructor of the service and explain nicely how settings can be passed from the server configuration file towards the service.

\lstsetARCHEDXML
\lstinputlisting
	[
	label=lst:echo_hed_configuration_xml, float=htb,
	caption={[Server configuration file for the echo service. Filename: arched\_echoservice.xml]
	\textbf{Server configuration file for the echo service. Filename: arched\_echoservice.xml}}
	]
{../src/services/echoservice/arched_echoservice.xml}





%\lstsetKSH
%\begin{lstlisting}[
%label=lst:invokation_arched_timeservice,float=htb,
%caption={[Transformation in eine uniforme konzentrische Verteilung.]
%         \textbf{Transformation in eine uniforme konzentrische Verteilung.\textcolor{white}{hmf}}}]
%$ arched -c arched_echoservice.xml  && echo jo ||echo n
%\end{lstlisting}



\section{Client}

With the interface of the server gaining complexity, so is the code for the client - for two reasons. Firstly, because the additional arguments need to be filled. Secondly, because of the extra effort to interact with the (here human) user to collect the data to forward to the Echo Web Service.
The parameters are now passed by the command line, to be seen in the lines followed by line~\ref{lst_code:echo_client_cpp_arguments}.
That code is independent from the ARC library.
The creation of the SOAP client is unchanged and the request is created in the lines subsequent line~\ref{lst_code:echo_client_cpp_request}.
It is processed in the code after line~\ref{lst_code:echo_client_cpp_response} in which also the response is received.
The checks of the response have expanded by an additional condition which tests for a fault message, see line~\ref{lst_code:echo_client_cpp_fault}.


\lstsetCPP
\lstinputlisting
	[
	label=lst:arcecho_arched_xml,
	caption={[HED configuration file for the ARC echo service. Filename: arcecho\_no\_ssl.xml]
	\textbf{HED configuration file for the ARC echo service. Filename: arcecho\_no\_ssl.xml\textcolor{white}{hmf}}}
	]
{../src/clients/echoclient/echoclient.cpp}

\lstsetCPP

The usage of the client is presented in Listing~\ref{lst:invokation_echoservice}.
Three cases of Echo Web Service responses are displayed: ordinary operation, reverse operation and SOAP fault due to an unknown operation.


\lstsetKSH
\begin{lstlisting}[
label=lst:invokation_echoservice, float=!htb,
caption={[Transformation in eine uniforme konzentrische Verteilung.]
         \textbf{Transformation in eine uniforme konzentrische Verteilung.\textcolor{white}{hmf}}}]
$ ./echoclient http://localhost:60000/echo ordinary text_to_be_transmitted
[ text_to_be_transmitted ]
$ ./echoclient http://localhost:60000/echo reverse text_to_be_transmitted
[ dettimsnart_eb_ot_txet ]
$ ./echoclient http://localhost:60000/echo re2verse text_to_be_transmitted
A SOAP fault occured:
  Fault code:   4
  Fault string: "Unknown operation. Please use "ordinary" or "reverse""
\end{lstlisting}

A few examples of exchanged messages are shown in the Listings~\ref{lst:echoservice_request}, \ref{lst:echoservice_response} and \ref{lst:echoservice_fault}. Listing~\ref{lst:echoservice_request} illustrates a request message using the operation \textit{reverse}.
Listing~\ref{lst:echoservice_response} illustrates the corresponding response message. When the operation requested by the client is unknown, then the server replies with the fault message shown in Listing~\ref{lst:echoservice_fault}.


\lstsetJUSTXML
\begin{lstlisting}[
label=lst:echoservice_request, float=!htb,
caption={[Transformation in eine uniforme konzentrische Verteilung.]
         \textbf{Transformation in eine uniforme konzentrische Verteilung.\textcolor{white}{hmf}}}]
<soap-env:Envelope xmlns:echo="urn:echo" xmlns:soap-enc="http://schemas.xmlsoap.org/soap/encoding/" xmlns:soap-env="http://schemas.xmlsoap.org/soap/envelope/" xmlns:xsd="http://www.w3.org/2001/XMLSchema" xmlns:xsi="http://www.w3.org/2001/XMLSchema-instance">
  <soap-env:Body>
    <echo:echoRequest>
      <echo:say operation="reverse">text_to_be_transmitted</echo:say>
    </echo:echoRequest>
  </soap-env:Body>
</soap-env:Envelope>
\end{lstlisting}



\lstsetJUSTXML
\begin{lstlisting}[
label=lst:echoservice_response, float=!htb,
caption={[Transformation in eine uniforme konzentrische Verteilung.]
         \textbf{Transformation in eine uniforme konzentrische Verteilung.\textcolor{white}{hmf}}}]
<soap-env:Envelope xmlns:echo="urn:echo" xmlns:soap-enc="http://schemas.xmlsoap.org/soap/encoding/" xmlns:soap-env="http://schemas.xmlsoap.org/soap/envelope/" xmlns:xsd="http://www.w3.org/2001/XMLSchema" xmlns:xsi="http://www.w3.org/2001/XMLSchema-instance">
  <soap-env:Body>
    <echo:echoResponse>
      <echo:hear>[ dettimsnart_eb_ot_txet ]</echo:hear>
    </echo:echoResponse>
  </soap-env:Body>
</soap-env:Envelope>
\end{lstlisting}


\lstsetJUSTXML
\begin{lstlisting}[
label=lst:echoservice_fault, float=!htb,
caption={[Transformation in eine uniforme konzentrische Verteilung.]
         \textbf{Transformation in eine uniforme konzentrische Verteilung.\textcolor{white}{hmf}}}]
<soap-env:Envelope xmlns:echo="urn:echo" xmlns:soap-enc="http://schemas.xmlsoap.org/soap/encoding/" xmlns:soap-env="http://schemas.xmlsoap.org/soap/envelope/" xmlns:xsd="http://www.w3.org/2001/XMLSchema" xmlns:xsi="http://www.w3.org/2001/XMLSchema-instance">
  <soap-env:Body>
    <soap-env:Fault>
      <soap-env:faultcode>soap-env:Client</soap-env:faultcode>
      <soap-env:faultstring>Unknown operation. Please use "ordinary" or "reverse"</soap-env:faultstring>
    </soap-env:Fault>
  </soap-env:Body>
</soap-env:Envelope>
\end{lstlisting}






