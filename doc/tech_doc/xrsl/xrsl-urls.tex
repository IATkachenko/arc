File locations in ARC can be specified both as local file
names, and as Internet standard \textit{Uniform Resource Locators
  (URL)}\index{URL}. There are also some additional URL
\textit{options} that can be used.

The following transfer protocols and metadata servers are supported:

\begin{tabular}{lp{10cm}}
   \texttt{ftp} & ordinary \textit{File Transfer Protocol (FTP)}\\
   \texttt{gsiftp} & GridFTP, the \globus\ -enhanced FTP protocol with
security, encryption, etc. developed by The Globus Alliance \cite{globus}\\
   \texttt{http} & ordinary \textit{Hyper-Text Transfer Protocol (HTTP)} with PUT and GET methods using multiple streams\\
   \texttt{https} & HTTP with SSL v3\\
   \texttt{httpg} & HTTP with \globus\  GSI\\
   \texttt{ldap} & ordinary \textit{Lightweight Data Access Protocol (LDAP)}~\cite{ldap}\\
   \texttt{rc} & \globus\  \textit{Replica Catalog (RC)}~\cite{rc}\\
   \texttt{rls} & \globus\  \textit{Replica Location Service (RLS)}~\cite{rls}\\
   \texttt{fireman} & Fireman indexing service of EGEE gLite~\cite{glite}\\
   \texttt{lfc} & LFC catalog and indexing service of EGEE gLite~\cite{glite}\\
   \texttt{se} & ARC Smart Storage Element service~\cite{se}\\
   \texttt{srm} & Storage Resource Manager (SRM) service~\cite{srm}\\
   \texttt{file} & local to the host file name with a full path\\
\end{tabular}

An URL can be used in a standard form, i.e.
\begin{shaded}
   \verb#<protocol>://host[:port]/<file>#
\end{shaded}

Or, to enhance the performance, it can have additional options:
\begin{shaded}
   \verb#<protocol>://host[:port][;option[;option[...]]]/<file>#
\end{shaded}

For a metadata service URL, construction is the following:
\begin{shaded}
   \verb#rc://rc://[location[|location[...]]@]<host>[:port]/<DN>/<lfn>#\\
   \verb#rls://[url[|url[...]]@]<host>[:port]/<lfn>#\\
   \verb#fireman://[url[|url[...]]@]<host>[:port]/<service_path>?<lfn>#\\
   \verb#lfc://[url[|url[...]]@]<host>[:port]/<lfn>#
\end{shaded}

For the Smart Storage Element service, the syntax is
\begin{shaded}
   \verb#se://host[:port][;options]/path[?file_id]#
\end{shaded}

For the SRM service, the syntax is
\begin{shaded}
   \verb#srm://<host>[:port][;options]/[service_path?SFN=]<file_id>#
\end{shaded}

Versions 1.1 and 2.2 of the SRM protocol are supported. The
default \emph{service\_path} is srm/managerv2 when the server supports
v2.2, srm/managerv1 otherwise.

The URL components are:

\begin{tabular}{lp{10cm}}
   \verb#location#&\verb#<location_name_in_RC>[;option[;option[...]]]#\\
   \verb#host[:port]#&IP address of a server\\
   \verb#DN#&Distinguished Name (as in LDAP) of an RC collection\\
   \verb#lfn#&Logical File Name\\
   \verb#url#&URL of the file as registered in RLS/Fireman\\
   \verb#service_path#&End-point path of the Web service\\
   \verb#file#&local to the host file name with a full path\\
\end{tabular}

The following options are supported for location URLs:\index{URL:options}

\begin{tabular}{lp{10cm}}
   \verb#threads=<number># & specifies number of parallel
   streams to be used by GridFTP or HTTP(s,g); default value is 1,
   maximal value is 10\\
   \verb#cache=yes|no|renew|copy# & indicates whether the GM should
   cache the file; default for input files is \verb#yes#. \verb#renew#
   forces a download of the file, even if the cached copy is still valid.
   \verb#copy# forces the cached file to be copied (rather than linked) to
   the session dir, this is useful if for example the file is to be modified.\\
   \verb#readonly=yes|no# & for transfers to \verb#file://# destinations,
   specifies whether the file should be
   read-only (unmodifiable) or not; default is \verb#yes# \\
   \verb#secure=yes|no# & indicates whether the GridFTP data
   channel should be encrypted; default is \verb#no#\\
   \verb#blocksize=<number># & specifies size of
   chunks/blocks/buffers used in GridFTP or HTTP(s,g) transactions;
   default is protocol dependent\\ 
   \verb#checksum=cksum|md5|adler32# & specifies the algorithm for checksum to be
   computed (ev. provided to the indexing server). This is overridden
   by any metadata options specified (see below).\\ 
   \verb#exec=yes|no# & means the file should be treated as executable\\
   \verb#preserve=yes|no# & specify if file must be uploaded to this
   destination even if job processing failed (default is \verb#no#)\\
   \verb#pattern=<pattern># & defines file matching pattern; currently
   works for file listing requests sent to an \verb#se://# endpoint\\
   \verb#guid=yes|no# & make software use GUIDs instead of LFNs while
   communicating to indexing services; meaningful for \verb#rls://#
   only\\
   \verb#overwrite=yes|no# & make software try to overwrite existing
   file(s), i.e. before writing to destination, tools will try to remove
   any information/content associated with specified URL\\
   \verb#protocol=gsi|gssapi# & to distinguish between two kinds of 
   \verb#httpg#. \verb#gssapi# stands for implemention using only GSSAPI 
   functions to wrap data and \verb#gsi# uses additional headers as 
   implmented in Globus IO\\
   \verb#spacetoken=<pattern># & specify the space token to be used for
   uploads to SRM storage elements supporting SRM version 2.2 or higher\\
   \verb#autodir=yes|no# & specify if before writing to specified location
   software should try to create all directories mentioned in specified
   URL. Currently this applies to FTP and GridFTP only. Default for those
   protocols is \verb#yes#\\
   \verb#tcpnodelay=yes|no# & controls the use of the TCP\_NODELAY
   socket option (which disables the Nagle algorithm). Applies to
   http(s) only. Default is \verb#no#\\ 
\end{tabular}

Local files are referred to by specifying either a location relative
to the job submission working directory, or by an absolute path (the
one that starts with "/"), preceded with a \verb#file://# prefix.

Metadata service URLs also support metadata options which can be used
for register additional metadata attributes or query the service using
metadata attributes. These options are specified at the end of the LFN
and consist of name and value pairs separated by colons. The following
attributes are supported:

\begin{tabular}{lp{10cm}}
   \verb#guid# & GUID of the file in the metadata service \\
   \verb#checksumtype# & Type of checksum. Supported values are cksum
   (default), md5 and adler32 \\
   \verb#checksumvalue# & The checksum of the file \\
\end{tabular}

Currently these metadata options are only supported for lfc:// URLs.

\begin{framed}
   Examples of URLs are:\\
   \\
   \verb#http://grid.domain.org/dir/script.sh#\\
   \verb#gsiftp://grid.domain.org:2811;threads=10;secure=yes/dir/input_12378.dat#\\
   \verb#ldap://grid.domain.org:389/lc=collection1,rc=Nordugrid,dc=nordugrid,dc=org#\\
   \verb#rc://grid.domain.org/lc=collection1,rc=Nordugrid,dc=nordugrid,dc=org/zebra/f1.zebra#\\
   \verb#rls://gsiftp://se.domain.org/datapath/file25.dat@grid.domain.org:61238/myfile02.dat#$^1$\\
   \verb#fireman://fireman_host:8443/glite-data-catalog-interface/FiremanCatalog?data.root#\\
   \verb#file:///home/auser/griddir/steer.cra#\\
   \verb#lfc://srm://srm.domain.org/griddir@lfc.domain.org//user/file1:guid=\# \\
   \verb#    bc68cdd0-bf94-41ce-ab5a-06a1512764dc:checksumtype=adler32:checksumvalue=12345678#$^2$\\
   \verb#lfc://;cache=no@lfc.domain.org/:guid=bc68cdd0-bf94-41ce-ab5a-06a1512764d#$^3$\\
\end{framed}

$^1$This is a destination URL. The file will be copied to the GridFTP
server at se.domain.org with the path datapath/file25.dat and
registered in the RLS indexing service at grid.domain.org with the LFN
myfile02.dat.

$^2$This is a destination URL. The file will be copied to
srm.domain.org at the path griddir/file1 and registered to the LFC
service at lfc.domain.org with the LFN /user/file1. The given GUID
and checksum attributes will be registered.

$^3$This is a source URL. The file is registered in the LFC service at
lfc.domain.org with the given GUID and can be copied or queried by
this URL. Note that as URL options are part of the location (physical)
URL, in meta service URLs the options must be part of the location
URL, even if the location URL is empty.
